%!TEX root = thesis.tex
%-------------------------------------------------------------------------------
\chapter{Local distinguishability in Matlab}
\label{chap:AppendixA}
%-------------------------------------------------------------------------------


\section*{Setup}
\subsection*{Requirements}

\begin{itemize}
    \item MATLAB\footnote{Unfortunately at the time of writing this, 
        the package CVX only runs on Matlab. I solemnly promise that I will port
        all the code to GNU Octave once the Octave port of CVX will be completed.};
    \item CVX  \textgreater= 2.1 \cite{cvx};
    \item QETLAB \textgreater= 0.7 \cite{Johnston2015};
\end{itemize}

\subsection*{List of functions}

\begin{itemize}
    \item \texttt{Distinguishability} (by N. Johnston) ---
        given an ensemble $\E$, computes $\opt(\E)$;
    \item \texttt{LocalDistinguishability} ---
        given an ensemble $\E$, computes $\opt_{\PPT}(\E)$ or $\opt_{\Sym}(\E)$;
    \item \texttt{UPBSepDistinguishable} (by N. Johnston) -- an implementation of
    the criterion described in Sec.~\ref{sec:criterion-sep-upb};
\end{itemize}

\sloppy
\definecolor{lightgray}{gray}{0.5}
\setlength{\parindent}{0pt}

\section*{Examples}

The code for the following examples can be found in the repository \cite{Cosentino15}. 

\subsection*{Yu--Duan--Ying states}

\begin{verbatim}
states = 1/2*[vec(kron(Pauli(0),Pauli(0)))'; ...
              vec(kron(Pauli(1),Pauli(1)))'; ...
              vec(kron(Pauli(2),Pauli(1)))'; ...
              vec(kron(Pauli(3),Pauli(1)))']';

disp(Distinguishability(states));
disp(LocalDistinguishability(states, 'copies', 1));
disp(LocalDistinguishability(states));
\end{verbatim}\color{lightgray} 
\begin{verbatim}     
    1
    0.8750
    0.7500
\end{verbatim} 

\color{black}

\subsection*{Tiles set plus extra orthogonal state (Example \ref{ex:tiles-set})}

\begin{verbatim}
extra_state = 1/2*[1 1 -1 0 0 -1 0 0 0];
set = [UPB('Tiles') extra_state'];

disp(LocalDistinguishability(UPB('Tiles')))
disp(LocalDistinguishability(states, 'copies', 1));
disp(LocalDistinguishability(states));
\end{verbatim}\color{lightgray} 
\begin{verbatim}    
    1.0000
    1.0000
    0.9860
\end{verbatim}

\color{black}

\subsection*{Bell state discrimination}

\begin{verbatim}
bells_m = [Bell(1) Bell(2) Bell(3) Bell(4)]';

disp(Distinguishability(bells_m));
disp(LocalDistinguishability(bells_m, 'copies', 1));
\end{verbatim}
\color{lightgray}
\begin{verbatim}     
    1
    0.5000
\end{verbatim} \color{black}
    
\subsection*{Entanglement cost (Section \ref{sec:entanglement-cost})} 

\begin{verbatim}
eps = 1/2; % takes a value between 0 and 1
eps_state = [sqrt((1+eps)/2) 0 0 sqrt((1-eps)/2)]';
\end{verbatim}


\subsubsection*{4 Bell states}

\begin{verbatim}
d = 4;
n = 4;

bells_m = zeros(d^2, n);

for k=1:n
    % makes sure we are considering the right partition
    bells_m(:, k) = Swap(kron(Bell(k), eps_state), [2 3], [2 2 2 2]);
end

disp(Distinguishability(bells_m));
disp(LocalDistinguishability(bells_m, 'copies', 1));
\end{verbatim}\color{lightgray}\begin{verbatim}     
    1
    0.9330
\end{verbatim} 
\color{black}
    
\subsubsection*{3 Bell states}

\begin{verbatim}
n = 3;
disp(Distinguishability(bells_m(:, 1:n)));
disp(LocalDistinguishability(bells_m(:, 1:n), 'copies', 1));
disp(LocalDistinguishability(bells_m(:, 1:n), 'copies', 2));
\end{verbatim}
\color{lightgray} 
\begin{verbatim}     
    1
    0.9916
    0.9586
\end{verbatim} \color{black}

\section*{Generalized Bell states (Examples from \cite{Ghosh04})}
In the code that follows, the function \texttt{GenPauli(a,b,n)} generates the 
matrix corresponding to the operator $W_{a,b}\in\Unitary(\complex^{n})$ 
defined in Eq.~\eqref{eq:generalized-Pauli-operators}.
\subsection*{$5$ states in $\complex^{5}\otimes\complex^{5}$}
\begin{verbatim}
n = 5;
gen_bells = [vec(GenPauli(0,0,n)), ...
             vec(GenPauli(1,1,n)), ...
             vec(GenPauli(1,2,n)), ...
             vec(GenPauli(3,1,n)), ...
             vec(GenPauli(3,2,n))];

disp(Distinguishability(gen_bells));
disp(LocalDistinguishability(gen_bells, 'copies', 1));
\end{verbatim}
\color{lightgray} 
\begin{verbatim}     
    1
    0.9898
\end{verbatim} \color{black}

\subsection*{$6$ states in $\complex^{6}\otimes\complex^{6}$}
\begin{verbatim}
n = 6;
gen_bells = [vec(GenPauli(0,0,n)), ...
             vec(GenPauli(1,1,n)), ...
             vec(GenPauli(0,2,n)), ...
             vec(GenPauli(0,3,n)), ...
             vec(GenPauli(0,4,n)), ...
             vec(GenPauli(3,0,n))];

disp(Distinguishability(gen_bells));
disp(LocalDistinguishability(gen_bells, 'copies', 1));
\end{verbatim}
\color{lightgray} 
\begin{verbatim}     
    1
    0.9905
\end{verbatim} \color{black}

