%!TEX root = thesis.tex
%----------------------------------------------------------------------
\chapter{Conclusions and open problems}
\label{chap:conclusions}
%----------------------------------------------------------------------

In this thesis we have used techniques from convex optimization to study 
the limitations of LOCC, separable, and PPT measurements for the task of 
distinguishing sets of bipartite states. Compared to previous approaches,
our techniques turned out to be effective in providing precise bounds on 
the maximum probability of locally distinguishing some interesting sets 
of maximally entangled states and unextendable product sets.

Several specific questions regarding the local distinguishability of sets of 
bipartite states remain unsolved.

In Chapter \ref{chap:mes} we proved a tight bound on the entanglement
cost of discriminating sets of Bell states by means of LOCC protocols. 
One could ask the following more general question.
\begin{question}
How much entanglement does it cost to distinguish 
maximally entangled states in $\complex^{n}\otimes\complex^{n}$?
\end{question}

Ghosh et al.~\cite{Ghosh04} have shown that orthogonal maximally
entangled states, which are in canonical form, can always be discriminated,
by means of LOCC protocols, if two copies of each of the states are provided.
One could ask if two copies are always sufficient. In fact, this question is open
even for separable and PPT measurement.
\begin{question}
Are two copies sufficient to discriminate any set of orthogonal pure states 
by PPT measurements?
\end{question}

In Chapter \ref{chap:mes} we have given an ensemble consisting only of 
orthogonal maximally entangled states for the task of distinguish which, 
PPT measurements are strictly more powerful than separable measurements.
One could inquire into a similar separation between LOCC and separable measurements.

\begin{question}
Does there exist an ensemble $\E$ fully composed of orthogonal maximally entangled 
states for which the following strict inequality holds?
  \begin{equation}
    \opt_{\LOCC}(\E) < \opt_{\Sep}(\E).
  \end{equation}
\end{question}

The techniques presented in the paper are not intrinsically limited 
to the setting of bipartite pure states, and applications of these techniques to
the \emph{multipartite} setting are topics for possible future work.

Global distinguishability of random states was studied by A. Montanaro \cite{Montanaro07}.
More precisely, he put a lower bound on the probability of distinguishing 
(by global measurements) an ensembles of $n$ random quantum states in $\complex^{d}$,
in the asymptotic regime where $n/d$ approaches a constant. A similar question 
on the distinguishability of random states by PPT and separable measurements 
could be investigated by using the convex optimization approach developed in the thesis.

Apart from quantum states, one could also study the LOCC distinguishability of 
quantum operations \cite{Matthews10}. It is a topic that has not been studied as throughly
as in the case of states, but once again, one could approach it through 
the lens of convex optimization.

A more speculative project, yet very exciting, is to give a somewhat useful 
characterization of the set dual to the set of LOCC measurement and its corresponding
set of linear mappings (both labeled by a question mark in the diagrams of Figure \ref{fig:measurements-dual}).
As we do not have a nice characterization of the LOCC set itself, we suppose
this is a difficult project. One direction to approach this problem
can be to consider smaller sets that are contained in the LOCC set, such as 
one-way LOCC, where the communication is only in one direction, say from
Alice to Bob, or LOCC-$r$ , in which the communication is limited to $r$ rounds.
Let us fantasize we had a characterization of the set dual to the set of LOCC measurement
and let us denote it as $\Meas_{\LOCC}^{\ast}(N, \X:\Y)$. Then we could plug it in the 
following cone program and proceed as we did for all the cone programs analyzed 
on this thesis.
\begin{center}
\underline{Dual (LOCC measurements)}
  \begin{equation}
    \label{eq:locc-dual-problem}
    \begin{split}
      \text{minimize:} \quad & \tr(H)\\
      \text{subject to:}
       \quad & 
        \begin{pmatrix}
            H - p_{1}\rho_{1} & & \\
             & \ddots & \\
             & & H - p_{N}\rho_{N}
       \end{pmatrix}\in \Meas_{\LOCC}^{\ast}(N, \X:\Y),\\
       \quad & H \in \Herm(\X\otimes\Y).
    \end{split}
  \end{equation}
\end{center}

Finally, apart from \cite{Gharibian13}, I am not aware of any results where 
cone programming is explicitly used in quantum computing for any cones different than the 
cone of semidefinite operators. 
I hope this work helps toward the rise of more applications of cone programming
in quantum information theory.
