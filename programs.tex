%!TEX root = thesis.tex
%-------------------------------------------------------------------------------
\chapter{A cone programming framework for local state distinguishability}
\label{chap:programs}
%-------------------------------------------------------------------------------

Due to the intrinsic complexity of LOCC protocols, it is hard 
to come up with techniques for their analysis.
This is true in particular for the analysis of the local state discrimination 
problem. All the proof techniques that have been proposed so far for this problem 
have their own limitations: they are mathematically cumbersome, or they bound the 
power only of limited subclasses of LOCC (one-way LOCC, for instance), or they 
can be applied only to very specific set of states. 

In this chapter we provide a more general framework based on convex optimization 
to prove bounds on LOCC protocols for the task of bipartite state discrimination.
We build on the idea described in Chapter \ref{chap:preliminaries} that LOCC 
measurements can be approximated by more tractable classes of measurements, 
in particular the sets of separable and PPT measurements.
It turns out that we can describe them conveniently using convex cones, 
and therefore, many problems in which we optimize over them can be cast 
into the cone programming paradigm. 

\minitoc

\section{General cone program}

The global state discrimination problem was one of the 
first applications of semidefinite programming to the theory of quantum 
information \cite{Eldar03a}.
Let us recall the parameters that define an instance of the problem. 
We are given a complex Euclidean space $\X$, a positive integer $N$, and an 
ensemble $\E$ of $N$ states, that is,
\begin{equation}
\label{eq:ensemble}
    \E = \big\{ (p_{1}, \rho_{1}), \ldots, (p_{N}, \rho_{N}) \big\},
\end{equation}
where $\rho_{1}, \ldots, \rho_{N} \in \Density(\X)$ and $(p_{1}, \ldots, p_{N})$ is
a probability vector.
We can construct a family of semidefinite programs parametrized by $\X$
and $N$, such that one program takes $\E \in \Ens(\X, N)$ as input and 
its optimal value corresponds to the maximum probability for any measurement to 
distinguish $\E$:
\begin{center}
\underline{Primal (Global measurements)}
\begin{equation}
  \label{eq:pos-primal-problem}
  \begin{split}
    \text{maximize:} \quad & 
    \sum_{k=1}^{N} p_{k}\ip{\rho_{k}}{\mu(k)},\\
    \text{subject to:} \quad & \sum_{k=1}^{N} \mu(k) = \I_{\X}\\
      & \mu : \{1,\ldots, N\}\rightarrow \Pos(\X)
  \end{split}
\end{equation}
\end{center}

The variables of the program form a collection of operators and the constraints 
impose that such collection of operators forms a valid measurements. 
In particular, the constraints demand that each operator belongs to the cone of 
semidefinite operators and that all the operators sum to identity, as in the
definition of measurement from Section \ref{sec:quantum-measurements}.

The key observation of this dissertation is that we can generalize the
above semidefinite program to a family of cone programs, where
the set of measurements over which we are optimizing forms a convex cone.
In effect, this generalization turns out to be helpful when the set of
measurements is characterized by the further property that each measurement in 
the set can be represented by restricting each of its measurement operators to 
belong to a particular convex cone.

More formally, say we are given a complex Euclidean space $\X$ and consider the
problem of distinguishing the ensemble $\E$ from  Eq. \eqref{eq:ensemble} by 
any measurement in some class 
\begin{equation}
  \K \subset \Meas(N,\X). 
\end{equation}
Further, suppose that the following characterization of $\K$ holds.
\begin{property}
\label{property:each-measurement-operator}
A measurement $\mu : \{1, \ldots, N\} \rightarrow \Pos(\X)$ belongs to the set $\K$ 
if and only if there exists a convex cone $\C \subset \Pos(\X)$ such that each 
measurement operator belongs to $\C$, that is, $\mu(k)\in \C$, 
for each $k \in \{1,\ldots, N\}$.
\end{property}

If this property is satisfied, the optimal probability of distinguishing $\E$ 
by any measurement in $\K$ is thus given by the optimal solution of the 
following cone program:
\begin{center}
\underline{Primal (General cone program)}
\begin{equation}
  \label{eq:cone-primal-problem}
  \begin{split}
    \text{maximize:} \quad & 
      \sum_{k=1}^{N} p_{k}\ip{\rho_{k}}{\mu(k)},\\
    \text{subject to:} \quad & \sum_{k=1}^{N} \mu(k) = \I_{\X}\\
    & \mu : \{1,\ldots, N\}\rightarrow \C
  \end{split}
\end{equation}
\end{center}

If one is to formally specify this problem according to the general form for
cone programs presented in Section \ref{sec:convex-optimization}, the function 
$\mu$ may be represented as a block matrix of the form
\begin{equation}
  X = \begin{pmatrix}
    \mu(1) & \cdots & \cdot \\
    \vdots & \ddots & \vdots\\
    \cdot & \cdots & \mu(N)
  \end{pmatrix} \in \Herm(\X \oplus \cdots \oplus \X)
\end{equation}
with the off-diagonal blocks being left unspecified.
The cone denoted by $\K$ in Section \ref{sec:cone-programming} is taken to be 
the cone of operators of this form for which each $\mu(k)$ belongs to the cone 
$\C$.

The mapping $\Phi$ and operators $A$ and $B$ 
are chosen in the natural way:
\begin{equation}
  A = \begin{pmatrix}
    p_1 \rho_1 & \cdots & 0 \\
    \vdots & \ddots & \vdots\\
    0 & \cdots & p_N\rho_N
  \end{pmatrix},
  \qquad
  B = \I_{\X},
\end{equation}
and $\Phi : \Lin(\X\oplus\cdots\oplus\X) \rightarrow \Lin(\X)$ is defined as 
\begin{equation}
  \Phi\begin{pmatrix}
  \mu(1) & \cdots & \cdot \\
  \vdots & \ddots & \vdots\\
  \cdot & \cdots & \mu(N)
  \end{pmatrix}
  \equiv \mu(1)+\cdots+\mu(N),
\end{equation}
for any $P_{1}, \ldots, P_{N} \in \Lin(\X)$.

Let $\Y = \complex^{N}$. One can easily verify that the mapping 
$\Phi^{\ast}: \Lin(\X)\rightarrow\Lin(\Y\otimes\X)$,
defined as  
\begin{equation}
  \Phi^{\ast}(H) \equiv \I_{\Y}\otimes H,
\end{equation}
satisfies Equation \eqref{eq:adjoint-map} and therefore is 
the adjoint of $\Phi$:
for any $H\in\Lin(\X)$. 
Also, let $\C^{\ast} \subset \Herm(\X)$ denote the dual cone of $\C$.
With these definitions in hand, one can write the dual of the Program 
\eqref{eq:cone-primal-problem} as follows:
\begin{center}
\underline{Dual (General cone program)}
\begin{equation}
  \label{eq:cone-dual-problem}
  \begin{split}
    \text{minimize:} \quad & \tr(H)\\
    \text{subject to:} \quad & H-p_k\rho_k\in\C^{\ast}
    \quad(\text{for each}\;k = 1,\ldots,N)\\
    \quad & H \in \Herm(\X).
  \end{split}
\end{equation}
\end{center}

Throughout the thesis we will use the property of \emph{weak duality} of cone 
programs (Proposition \ref{prop:weak-duality-cone}) to upper bound the
optimal solution of the primal program \eqref{eq:cone-primal-problem}.
The cone programs considered in this thesis also possess the property 
of \emph{strong duality}. This property depends on the specific cone $\C$,
and we will discuss it whenever we treat a specific $\C$, although
it should be noted that strong duality is not needed for any of our results.

In the rest of this chapter, we will see different instantiations of the general
program for a variety of measurements classes. 
We started this section by presenting the semidefinite program 
\eqref{eq:pos-primal-problem} for the problem of  
state distinguishability by global measurement. In that case, the cone $\C$ 
of the general cone program 
corresponded to the cone of semidefinite operators, that is, $\C = \Pos(\X)$ and, 
due to Fact \ref{fact:pos-self-dual}, we have that $\C^{\ast} = \C$. 
Thus we can write the following dual program of \eqref{eq:pos-primal-problem}:
\begin{center}
\underline{Dual (Global measurements)}
\begin{equation}
  \label{eq:global-dual-problem}
  \begin{split}
    \text{minimize:} \quad & \tr(H)\\
    \text{subject to:} \quad & H-p_k\rho_k \in \Pos(\X)
    \quad(\text{for each}\;k = 1,\ldots,N)\\
    \quad & H \in \Herm(\X).
  \end{split}
\end{equation}
\end{center}


\section{Bipartite measurements}
The above generalization of the optimal measurement cone program turns out to be
particularly helpful for the analysis of the bipartite state discrimination problem,
which is the main focus of this thesis.

Recall that, as input of the problem, we are given two complex Euclidean 
spaces $\X$ and $\Y$, one for each party, a positive integer $N$, and an 
ensemble of states that are distributed among the spaces of the two parties, 
that is,
\begin{equation}
\label{eq:ensemble-bipartite}
    \E = \big\{ (p_{1}, \rho_{1}), \ldots, (p_{N}, \rho_{N}) \big\},
\end{equation}
with $\rho_{1}, \ldots, \rho_{N} \in \Density(\X\otimes\Y)$.

Ideally, we would like to solve the following problem:
\begin{center}
\underline{Primal (LOCC measurements)}
  \begin{equation}
    \label{eq:locc-primal-problem}
    \begin{split}
      \text{maximize:} \quad & 
        \sum_{k=1}^{N} p_{k}\ip{\rho_{k}}{\mu(k)},\\
      \text{subject to:} %\quad & \sum_{k=1}^{N} P_{k} = \I_{\X\otimes\Y}\\
       \quad & \mu \in \Meas_{\LOCC}(N, \X:\Y).
    \end{split}
  \end{equation}
\end{center}

Phrasing this problem as a cone program is not interesting as such. Even though it is
technically possible, we would not be able to exploit the advantages that come from such formulation, 
due to the fact (by now, we have stressed this enough!) that the set of LOCC measurements 
is not easy to be handled mathematically.
For the set of LOCC measurement, we do not even have a characterization
on the same lines of Property \ref{property:each-measurement-operator}, so we cannot cast
the problem in the general form of the program in \eqref{eq:cone-primal-problem}. 

As indicated in Chapter \ref{chap:bipartite-state-discrimination}, the LOCC set
can be approximated by other sets of measurements that are easier to be manipulated 
mathematically. It turns out that both the sets of PPT and separable measurements are
suitable for the cone programming framework described above. In fact, for both 
these sets it is relatively easy to characterize the dual set, and an equivalent 
of Property \ref{property:each-measurement-operator} holds,
For example, Proposition \ref{prop:separable-each} characterizes a separable 
measurement over the bipartition $(\X:\Y)$ as a collection of operators 
belonging to the cone $\Sep(\X:\Y)$.
This property allows us to characterize the maximum probability of 
distinguishing the ensemble in Eq.~\eqref{eq:ensemble-bipartite} by any 
separable measurement as a cone program of the same form as the one 
in~\eqref{eq:cone-primal-problem}, where instead of $\X$, the underlying space 
of the operators is $\X\otimes\Y$, and $\C(\X)$ is replaced by the cone of 
separable operators $\Sep(\X:\Y)$.

In the rest of this section, we study the different programs that derive 
from the Program~\eqref{eq:cone-primal-problem} when we instantiate $\C$ with
some particular cones corresponding to different classes of bipartite 
measurements. In particular, we will mainly look at the programs derived from the cones
of separable operators, PPT operators, and operators with $k$-symmetric extensions.
For each of these programs, we show the dual program and try to make any possible
simplification. Moreover, we point out whenever a program can be 
expressed by using only semidefinite constraints, as it was the case for the 
Program~\eqref{eq:pos-primal-problem} from above.

\subsection{PPT measurements}

We start with the cone of PPT operators and we describe a semidefinite program 
that computes $\opt_{\PPT}(\E)$. Using tools of convex optimization to solve 
problems concerning the PPT cone is not a novel idea. 
Two other applications of convex programming to the realm of PPT operations 
are the semidefinite program shown by Rains to compute the maximum fidelity 
obtained by a PPT distillation protocol \cite{Rains01} and the hierarchy of 
semidefinite programs proposed as separability criteria by Doherty, Parrilo, and
Spedalieri \cite{Doherty02,Doherty04}.

Recall from Definition \ref{def:ppt-measurements} that a measurement 
$\mu : \{1, \ldots, N\} \rightarrow \Pos(\X\otimes\Y)$ is in 
$\Meas_{\PPT}(N, \X:\Y)$ if and only if
\begin{equation}
  \mu(1), \ldots, \mu(N) \in \PPT(\X:\Y).
\end{equation}
From the definition of $\PPT(\X:\Y)$, we can write the cone program in
the following form:
\begin{center}
\underline{Primal (PPT measurements)}
\begin{equation}
  \label{eq:ppt-primal-problem}
  \begin{split}
    \text{maximize:} \quad & 
      \sum_{k=1}^{N} p_{k}\ip{\rho_{k}}{\mu(k)},\\
    \text{subject to:} \quad & \sum_{k=1}^{N} \mu(k) = \I_{\X\otimes\Y}\\
    & \mu : \{1,\ldots, N\}\rightarrow \Pos(\X\otimes\Y),\\
    & \pt_{\X}(\mu(k))\in\Pos(\X\otimes\Y) \quad(\text{for each}\;k = 1,\ldots,N).
  \end{split}
\end{equation}
\end{center}
An immediate observation is that the cone program above is in fact a semidefinite 
program. To see this formally, let us introduce $N$ variables 
$Q_{1},\ldots,Q_{N} \in \Herm(\X\otimes\Y)$ and, for each $k \in \{1, \ldots, N\}$, let
\begin{equation}
  Q_{k} = \pt_{\X}(\mu(k)).
\end{equation}
One can write the above program as a semidefinite program in the standard form of 
Section~\ref{sec:semidefinite-programming}, where
\begin{equation}
\label{eq:ppt-X}
X = \begin{pmatrix}
      \mu(1) & \cdots & \cdot\\
      \vdots & \ddots & \vdots\\
      \cdot & \cdots & \mu(N)\\
  \end{pmatrix}
  \oplus
  \begin{pmatrix}
      Q_1 & \cdots & \cdot\\
      \vdots & \ddots & \vdots\\
      \cdot & \cdots & Q_N\\
  \end{pmatrix}
\end{equation}
is the variable over which we optimize,
\begin{equation}
  A = \begin{pmatrix}
      p_{1}\rho_{1} & \cdots & \cdot\\
      \vdots & \ddots & \vdots\\
      \cdot & \cdots & p_{N}\rho_{N}\\
  \end{pmatrix}
  \oplus
  \begin{pmatrix}
      0 & \cdots & \cdot\\
      \vdots & \ddots & \vdots\\
      \cdot & \cdots & 0\\
  \end{pmatrix},
\end{equation}
and
\begin{equation}  
  B = 
  \begin{pmatrix}
      \I_{\X\otimes\Y} & \cdot & \cdots & \cdot \\
        \cdot  & 0 & \cdots & \cdot\\
        \vdots & \vdots & \ddots & \vdots\\
        \cdot  & \cdot & \cdots & 0
    \end{pmatrix}.
\end{equation}
are the known inputs of the problem, and the map $\Phi : $ is defined as 
\begin{equation}
  \Phi(X) \equiv \begin{pmatrix}
      \mu(1) + \cdots +\mu(N) & \cdot & \cdots & \cdot \\
        \cdot  & \pt_{\X}(\mu(1)) - Q_{1} & \cdots & \cdot\\
        \vdots &\vdots & \ddots & \vdots\\
        \cdot  &\cdot & \cdots & \pt_{\X}(\mu(N)) - Q_{N}
    \end{pmatrix},
\end{equation}
for any operator $X$ in the form of Eq.~\eqref{eq:ppt-X}.

Once the program is in the standard form, one can easily derive its dual.
The variable of the dual program is the Hermitian operator
$Y \in $ 
defined as follows:
\begin{equation}
\label{eq:ppt-Y}
  Y = 
  \begin{pmatrix}
      H & \cdot & \cdots & \cdot \\
        \cdot  & -R_{1} & \cdots & \cdot\\
        \vdots & \vdots & \ddots & \vdots\\
        \cdot  & \cdot & \cdots & -R_{N}
    \end{pmatrix},
\end{equation}
for Hermitian operators $Y, R_{1}, \ldots, R_{N} \in \Herm(\X\otimes\Y)$.
The adjoint of $\Phi$ is defined as the mapping
\begin{equation}
  \Phi^{\ast}(Y) \equiv 
    \begin{pmatrix}
      H - \pt_{\X}(R_{1}) & \cdots & \cdot\\
      \vdots & \ddots & \vdots\\
      \cdot & \cdots & H - \pt_{\X}(R_{N})\\
    \end{pmatrix}
    \oplus
    \begin{pmatrix}
      R_{1} & \cdots & \cdot\\
      \vdots & \ddots & \vdots\\
      \cdot & \cdots & R_{N}\\
    \end{pmatrix},
\end{equation}
for any operator $Y$ in the form of Eq.~\eqref{eq:ppt-Y}.
It is easy to verify that the map $\Phi^{\ast}$ satisfies Eq.~\eqref{eq:adjoint-map}.
From the fact the the partial transpose is its own adjoint and inverse, we have 
that
\begin{equation}
  \ip{A}{B} = \ip{\pt_{X}(A)}{\pt_{X}(B)},
\end{equation}
for any operators $A, B\in\Lin(\X\otimes\Y)$, which implies
\begin{equation}
\ip{\mu(k)}{\pt_{\X}(R_{k})} = \ip{\pt_{\mu(k)}}{R_{k}},
\end{equation}
for any $k \in \{1, \ldots, N\}$, and therefore
\begin{equation}
  \ip{Y}{\Phi(X)} = \ip{\Phi^{\ast}(Y)}{X}.
\end{equation}
By rearranging everything in a more explicit form, we have the following dual program:
\begin{center}
\underline{Dual (PPT measurements)}
\begin{equation}
  \label{eq:ppt-dual-problem}
  \begin{split}
    \text{minimize:} \quad & \tr(H)\\
    \text{subject to:} \quad & H-p_k\rho_k \geq \pt_{\X}(R_{k})
    \quad(\text{for each}\;k = 1,\ldots,N),\\
    \quad & R_{1}, \ldots, R_{N} \in \Pos(\X\otimes\Y),\\
    \quad & H \in \Herm(\X\otimes\Y).
  \end{split}
\end{equation}
\end{center}

\subsubsection{Decomposable operator}
An equivalent way of deriving the above dual program is by defining the cone 
\begin{equation}
  \PPT^{\ast}(\X:\Y) = \{ S + \pt_{\X}(R) \,:\, S, R \in \Pos(\X\otimes\Y) \},
\end{equation}
which satisfies \eqref{eq:dual-cone} and therefore is the dual cone of 
$\PPT(\X:\Y)$.
The program above corresponds to an instance of 
the generic dual program \eqref{eq:cone-dual-problem}, where $\C^{\ast}$ is 
replaced by $\PPT^{\ast}(\X:\Y)$.

The operators in $\PPT^{\ast}(\X:\Y)$ can also be characterized as 
representations of so-called \emph{decomposable maps} from 
$\Lin(\Y)$ to $\Lin(\X)$, via the Choi isomorphism 
(see Section \ref{sec:choi-isomorphism})
\begin{definition}
A decomposable map $\Phi:\Lin(\Y)\rightarrow\Lin(\X)$ is a linear map that can 
be represented as the sum of a completely positive map and a completely 
co-positive map, that is, there exist two completely positive maps 
$\Psi,\Xi:\Lin(\Y)\rightarrow\Lin(\X)$, such that, for any $Y\in\Lin(\Y)$,
\begin{equation}
  \Phi(Y) = \Psi(Y) + (\pt {\circ}\, \Xi)(Y),
\end{equation}
where $\pt$ denotes the transpose map.
\end{definition}

\subsubsection{Exploiting symmetries}

In cases where the ensemble of states we wish to distinguish exhibit some symmetry,
we can simplify the semidefinite program ($\PP_{\PPT}$) to a linear program.
One particular case where this kind of symmetry emerges is when 
we consider so-called \emph{lattice states}. Let
\[
  \psi_{i} = \ket{\psi_{i}}\bra{\psi_{i}} \in \Density(\complex^{2}\otimes\complex^{2}),
\]
for $i\in \{0,1,2,3\}$, be the density operators corresponding to the standard Bell 
states, as defined in Eq.~\eqref{eq:Bell-states}.
Let $v \in \integer_{4}^{t} $ be a $t$-dimensional vector and let 
$\ket{\psi_{v}} \in \complex^{2^{t}}\otimes\complex^{2^{t}}$ 
be the maximally entangled state given by the tensor product of Bell states indexed by the vector 
$v = (v_{1}, \ldots, v_{t})$, that is,
$$
\ket{\psi_{v}} = \ket{\psi_{v_{1}}}\otimes\ldots\otimes\ket{\psi_{v_{t}}}.
$$
In the literature, operators diagonal in the basis 
$\{\psi_{v} = \ket{\psi_{v}}\bra{\psi_{v}} : v \in \integer_{4}^{t}\}$
are called \emph{lattice operators}, or \emph{lattice states} if they are also 
density operators \cite{Piani06}.

The following equations regarding the partial transpose of the Bell states will 
be used in the main proof of this section and can be proved by direct inspection.
\begin{equation}
  \label{eq:ppt-bell-states}
  \begin{aligned}
    \pt_{\X}(\psi_{0}) = \frac{1}{2}\I - \psi_{2},\qquad
    \pt_{\X}(\psi_{1}) = \frac{1}{2}\I - \psi_{3},\\
    \pt_{\X}(\psi_{2}) = \frac{1}{2}\I - \psi_{0},\qquad
    \pt_{\X}(\psi_{3}) = \frac{1}{2}\I - \psi_{1}.
  \end{aligned}
\end{equation}
The following proposition is useful for the proof of the main theorem of this 
section and, again, it can be easily proved by direct inspection.
\begin{prop}
\label{prop:groupG}
Let $\{\sigma_{0}, \sigma_{1}, \sigma_{2}, \sigma_{3}\} \subset \Herm(\complex^{2})$ 
be the set of Pauli operators in defined in Eq.~\eqref{eq:Pauli-operators}.
It holds that the Bell states from Eq.~\eqref{eq:Bell-states} are invariant under the group of local symmetries
\begin{equation}
  G = \big\{ \sigma_{i} \otimes \sigma_{i} : i\in\{0,1,2,3\}  \big\},
\end{equation}
that is, $\psi_{i} = U\psi_{i}U^{*}$ for any $U \in G$ and any $i\in\{0,1,2,3\}$.
\end{prop}

It turns out that in the case when the set to distinguish contains only lattice states,
the semidefinite program ($\PP_{\PPT}$) simplifies remarkably, as it is established 
by the following theorem.
\begin{theorem}
If the set to be distinguished consists only of lattice states, then the probability of 
successfully distinguishing them by PPT measurements can be expressed as the 
optimal value of a linear program.
\end{theorem}
\begin{proof}
We will prove that for any feasible solution of the semidefinite program ($\PP_{\PPT}$), 
there is another feasible solution consisting only of lattice operators 
for which the objective function takes the same value.

Let $\Delta : \Lin(\complex^{2}\otimes\complex^{2}) \rightarrow 
  \Lin(\complex^{2}\otimes\complex^{2})$ be the channel defined as follows:
\begin{equation}
  \Delta(X) = \frac{1}{|G|}\sum_{U \in G} UXU^{*},
\end{equation}
where $G$ is the group of local unitaries defined in Proposition \ref{prop:groupG}. 
The channel $\Delta(X)$ acts on $X$ as a completely dephasing channel in the Bell basis.
Let $\X^{(t)} = \Y^{(t)} = \complex^{2^{t}}$ for some positive integer $t > 1$. 
Say the states we want to distinguish,
\begin{equation}
  \rho_{1}, \ldots, \rho_{N} \in \Density(\X^{(t)}\otimes\Y^{(t)}),
\end{equation}
are lattice states.
Let $\Phi = \Delta^{\otimes t}$ be the $t$-fold tensor product of the map $\Delta$,
that is,
\begin{equation}
  \Phi(v_{1}\otimes\cdots\otimes v_{t}) = 
    \Delta(v_{1})\otimes\cdots\otimes\Delta(v_{t}),
\end{equation}
for any choice of vectors $v_{1}, \ldots, v_{t} \in \X\otimes\Y$. 
Assume that a measurement 
\[
  \mu : \{1,\ldots,N\}\to\PPT(\X:\Y)
\] 
is a feasible solution of the program ($\PP_{\PPT}$) 
for the states $\rho_{1}, \ldots, \rho_{N}$. 
In the rest of the proof we want to show that a new measurement $\mu'$ constructed by 
applying $\Phi$ to each measurement operator of $\mu$ is also a feasible solution
of the program ($\PP_{\PPT}$), for the same set of states.
Since $\Phi$ is a dephasing channel in the lattice basis, this would imply the 
statement of the theorem. 

First, let us observe that the value of the objective function for the solution 
$\mu'$ is the same as the value for the original solution $\mu$.
The channel $\Phi$ is its own adjoint and therefore we have
\[
  \ip{\mu(k)}{\rho_{k}} = \ip{\mu(k)}{\Phi(\rho_{k})} = 
  \ip{\Phi(\mu(k))}{\rho_{k}} ,
\]
for any $k = 1, \ldots, N$.

Next, we show that $\mu'$ is a PPT measurement. From the fact that $\Phi$ is unital 
(in fact it is a mixed unitary channel), we have
\begin{equation}
  \Phi(\mu(1)) + \ldots + \Phi(\mu(N)) = \I,
\end{equation}
and from the fact that $\Phi$ is positive, we have 
\begin{equation}
  \Phi(\mu(k)) \in\Pos(\X^{(t)}\otimes\Y^{(t)})
\end{equation}
and
\begin{equation}
\label{eq:pt-mu}
  \Phi(\pt_{\X}(\mu(k))) \in\Pos(\X^{(t)}\otimes\Y^{(t)}),
\end{equation}
for any $k \in \{1, \ldots, N \}$. 
The first fact implies that $\mu'$ is indeed valid measurement.
The second fact is close to what we want in order to show that $\mu'$ is a PPT measurement.
To complete the proof, we wish to show that the partial transpose mapping commutes 
with the channel $\Phi$.
First we observe how the partial transposition modifies the action of local operators.
Given linear operators $A\in \Lin(\X)$, $B \in \Lin(\Y)$ and $X \in \Lin(\X \otimes \Y)$, 
we have
\begin{equation}
  \pt_{\X} [(A \otimes B)X(A \otimes B)^{*}] = 
  (\overline{A}\otimes B)\pt_{\X}(X)(\overline{A}\otimes B)^{*}.
\end{equation}
Now, for the Pauli matrices, we have $\overline{\sigma_{j}} = \sigma_{j}$ 
for $j \in \{ 0,1,3\}$ and $\overline{\sigma_{2}} = -\sigma_{2}$. 
Therefore 
\begin{equation}
  \Delta(\pt_{\X}(X)) = \pt_{\X}(\Delta(X)), \quad \text{for any $X \in \Lin(\X\otimes\Y)$}.
\end{equation}
This property trivially extends by tensor product to $\Phi$ and therefore 
Eq.~\eqref{eq:pt-mu} implies
\begin{equation}
  \pt_{\X}(\Phi(\mu(k))) \geq 0, 
\end{equation}
for any $k \in \{1, \ldots, N\}$, which concludes the proof.
\end{proof}

The advantage of the linear programming formulation is in the computational 
efficiency of the algorithms that solve the program.
However, for sake of clarity, in the analytic proofs that will follow, 
we will always stick to the more general semidefinite programming formulation, 
even when we consider distinguishability of lattice states.

\subsection{Separable measurements}
We have yet to fully exploit the expressive power of the cone programming language.
In this section we do so by showing a connection between convex optimization and
the cone of separable operators, which would not be possible if the only tool at 
hand was semidefinite programming. 
Surprisingly, there are only few examples in the literature (\cite{Gharibian13}
being one of those) where this connection has been made use of.

As it is stated by Proposition \ref{prop:separable-each}, a measurement
$\mu : \{1,\ldots,N\} \rightarrow \Pos(\X\otimes\Y)$ is separable when 
\begin{equation}
  \mu(1), \ldots, \mu(N) \in \Sep(\X:\Y).
\end{equation}
We can write the maximum probability of distinguishing an ensemble $\E$ by 
separable measurements, $\opt_{\Sep}(\E)$, as the optimal value of the 
following cone program: 
\begin{center}
\underline{Primal ($\PP_{\Sep}$)}
\begin{equation}
  \label{eq:sep-primal-problem}
  \begin{split}
    \text{maximize:} \quad & 
      \sum_{k=1}^{N} p_{k}\ip{\rho_{k}}{\mu(k)},\\
    \text{subject to:} \quad & \sum_{k=1}^{N} \mu(k) = \I_{\X\otimes\Y},\\
        & \mu : \{1,\ldots, N\}\rightarrow \Sep(\X:\Y).
  \end{split}
\end{equation}
\end{center}

An important observation about this program is that, unlike the case of the 
PPT program, its constraints cannot be formulated as semidefinite constraints.
% \comment{AC}{``cannot be formulated'' is not a formal statement, make this formal}
Later in Section \ref{sec:computational-aspects} we will see how this has 
implications in the computational complexity of the problem.

We denote the cone dual to $\Sep(\X:\Y)$ by $\BPos(\X:\Y)$, and defer its definition
to the next paragraph, after we write the dual program of \eqref{eq:sep-primal-problem}:
\begin{center}
\underline{Dual ($\DP_{\Sep}$)}
\begin{equation}
  \label{eq:sep-dual-problem}
  \begin{split}
    \text{minimize:} \quad & \tr(H)\\
    \text{subject to:} \quad & H-p_k\rho_k\in\BPos(\X:\Y)
    \quad(\text{for each}\;k = 1,\ldots,N),\\
    \quad & H \in \Herm(\X\otimes\Y).
  \end{split}
\end{equation}
\end{center}

\subsubsection{Block-positive operators}
\label{sec:block-positive-operators}

The cone $\BPos(\X : \Y)$, which is commonly known as the set of 
\emph{block-positive operators}, is
\begin{equation}
  \BPos(\X:\Y) = \bigl\{
  H\in\Herm(\X\otimes\Y)\,:\,
  \ip{P}{H} \geq 0 \;\text{for every $P \in \Sep(\X:\Y)$}\bigr\}.
\end{equation}
There are several equivalent characterizations of this set.
For instance, one has
\begin{multline}
  \BPos(\X:\Y) = \bigl\{
  H\in\Herm(\X\otimes\Y)\,:\\
  (\I_{\X} \otimes y^{\ast}) H (\I_{\X} \otimes y)\in\Pos(\X)
  \;\text{for every $y\in\Y$}\bigr\}.
\end{multline}
Alternatively, block-positive operators can be characterized as representations 
of \emph{positive} linear maps, via the Choi representation.
That is, for any linear map $\Phi:\Lin(\Y)\rightarrow\Lin(\X)$
mapping arbitrary linear operators on $\Y$ to linear operators on $\X$, the
following two properties are equivalent:
\begin{itemize}
\item[(a)] For every positive semidefinite operator $Y \in \Pos(\Y)$, it 
holds that $\Phi(Y) \in \Pos(\X)$. 
\item[(b)] The Choi operator
\begin{equation}
  J(\Phi) = \sum_{0\leq j,k< m} \Phi\bigl(\ket{j}\bra{k}\bigr) 
  \otimes \ket{j}\bra{k}
\end{equation}
of $\Phi$ satisfies $J(\Phi) \in \BPos(\X:\Y)$.
\end{itemize}

Dual to the fact that there are positive semidefinite operators, which are not 
separable, is the fact that there are block-positive operators, which are not 
positive semidefinite. Such operators are called \emph{entanglement witnesses} 
and represent (via the Choi representation) linear maps that are positive, 
but not completely positive.

One example of entanglement witness is the swap operator defined in 
\eqref{eq:swap-operator}, which is the Choi representation
of the transpose map.
We will see many other example of entanglement witnesses in later sections;
for a more exhaustive review, see \cite{Chruscinski2014}.

The Venn diagram in Figure \ref{fig:measurements-dual} depicts the containments
of sets that are dual to the sets of measurements operators we have considered  
(a set with one shade of gray is dual to the set with the same shade of gray from
Figure \ref{fig:classes-measurements}).

\begin{figure}[!ht]
  \centering
    \begin{minipage}{0.5\textwidth}
      \centering
      \def\svgwidth{200pt}
      \scalebox{.75}{\input{drawing_dual.pdf_tex}}
    \end{minipage}\hfill
    \begin{minipage}{0.5\textwidth}
      \centering
      \def\svgwidth{200pt}
      \scalebox{.75}{\input{drawing_super.pdf_tex}}
    \end{minipage}
    \caption{Sets of operators that are dual to the sets of Figure 
    \ref{fig:classes-measurements} (on the left) and the 
    corresponding sets of linear mappings via the Choi isomorphism (on the right).}
    \label{fig:measurements-dual}
\end{figure}

\subsection{Symmetric extensions}
\label{sec:symm-ext}
For any given ensemble of states $\E$, the cone program \eqref{eq:sep-primal-problem}
for separable measurements outputs a better (not always strictly better) approximation of $\opt_{\LOCC}(\E)$, 
compared to the output of the semidefinite program \eqref{eq:ppt-primal-problem} for PPT measurements.
The drawback is in the computational complexity: on the one extreme, we have 
polynomial time algorithms to solve the PPT semidefinite program, and on the other 
extreme, we can prove that optimizing over a set of separable operator is an NP-hard 
problem (more details in Section \ref{sec:computational-aspects}).

Building up on an idea by Doherty, Parrilo, and Spedalieri \cite{Doherty02,Doherty04}, 
we are able to interpolate between these two extremes and construct a hierarchy of
semidefinite programs characterized by the following trade-off: whenever we 
climb up a level of the hierarchy, the size of the program increases 
(and so does the running time of the algorithms that solve it), however the
outputs of the program gives an approximation closer to $\opt_{\Sep}(\E)$.

In order to formally describe the hierarchy of semidefinite programs that 
approximates $\opt_{\Sep}(\E)$, we first need to introduce the concept of 
symmetric extension of a positive operator. 
First let us define the set of permutation operators.
Let $s$ be a positive integer and let $\X_{1}, \ldots, \X_{s}$ be $s$ isomorphic 
copies of the same complex Euclidean space, that is, for some positive integer
$d$ and any $k \in \{1, \ldots, m \}$, $\X_{k} \simeq \complex^{d}$. 
Given a permutation $\pi\in\Sym(s)$, we define
a \emph{permutation operator} 
\begin{equation}
  W_{\pi} \in \Unitary(\X_{1}\otimes\cdots\otimes\X_{s})
\end{equation}
to be the unitary operator that acts as follows:
\begin{equation}
  W_{\pi}(u_{1}\otimes\cdots\otimes u_{s}) = u_{\pi(1)}\otimes\cdots
    \otimes u_{\pi(s)},
\end{equation}
for any choice of vectors $u_{1}, \ldots, u_{s}\in\complex^{d}$.
We are now ready to give the definition of symmetric extension of a positive 
semidefinite operator. 

\begin{definition}
  Suppose we are given complex Euclidean spaces $\X$ and $\Y$, and an operator
  $P\in\Pos(\X\otimes\Y)$. Moreover, let $s$ be a positive integer and let 
  $\Y_{2},\ldots,\Y_{s}$ be copies of the space $\Y$. An operator 
  \begin{equation}
    X \in \Pos(\X\otimes\Y\otimes\Y_{2}\otimes\cdots\otimes\Y_{s})
  \end{equation}
  is called an \emph{$s$-symmetric extension} of $P$ if the following two 
  properties hold:
  \begin{itemize}
    \item[(a)] $\tr_{\Y_{2}\otimes\cdots\otimes\Y_{s}}(X) = P$;
    \item[(b)] $(\I_{\X}\otimes W_{\pi})X(\I_{\X}\otimes W_{\pi}^{\ast}) = X$,
      for every $\pi\in\Sym(s)$.
  \end{itemize}
\end{definition}

Any separable operator possesses $s$-symmetric extensions, for any $s \geq 1$. 
This is easy to see from the definition of separable operator: given 
$P\in\Sep(\X:\Y)$, there exists a positive integer $M$, positive semidefinite 
operators $Q_1,\ldots,Q_M \in \Pos(\X)$ and density matrices $\rho_1,\ldots, \rho_M\in\Density(\Y)$ 
such that
\begin{equation}
  P = \sum_{k = 1}^M Q_{k} \otimes \rho_{k}.
\end{equation} 
Then, for any $s \geq 1$,
\begin{equation}
\label{eq:X-extension}
  X = \sum_{k = 1}^M Q_k \otimes \rho_{k}^{\otimes s}
\end{equation}
is an $s$-symmetric extension of $P$.

Interestingly, the converse is also true, that is, for any entangled operator 
\begin{equation}
P \in \Pos(\X:\Y) \setminus \Sep(\X:\Y),
\end{equation}
there must exist a value $s > 1$ for which $P$ does not have a $s$-symmetric 
extension. For a proof of this fact, which is more involved, we refer the 
reader to the original paper \cite{Doherty02}.



It is worthwhile to consider two additional constraints we can add to the 
definition of symmetric extension. 
The first one comes from the observation that the symmetric part of the operator
$X$ in Eq.~\eqref{eq:X-extension} is supported by the symmetric subspace, which
is defined as the set of all vectors in $\Y\otimes\Y_{2}\otimes\cdots\otimes\Y_{s}$
that are invariant under the action of $W_{\pi}$, for every choice of $\pi\in\Sym(s)$.
We denote the symmetric subspace by
\begin{equation}
  \Y\ovee\Y_{2}\ovee\cdots\ovee\Y_{s} = \{ y \in \Y\otimes\Y_{2}\otimes\cdots\otimes\Y_{s}
    : y = W_{\pi}y \;\mbox{ for every $\pi\in\Sym(s)$} \},
\end{equation}
and the projection on this subspace by $\Pi_{\Y\ovee\Y_{2}\ovee\cdots\ovee\Y_{s}}$.

\begin{definition}
\label{def:bosonic}
  Suppose we are given complex Euclidean spaces $\X$ and $\Y$, and an operator
  $P\in\Pos(\X\otimes\Y)$. Moreover, let $s$ be a positive integer and let 
  $\Y_{2},\ldots,\Y_{s}$ be copies of the space $\Y$. An operator 
  \begin{equation}
    X \in \Pos(\X\otimes\Y\otimes\Y_{2}\otimes\cdots\otimes\Y_{s})
  \end{equation}
  is called an \emph{$s$-symmetric bosonic extension} of $P$ if the following 
  two properties hold:
  \begin{itemize}
    \item[(a)] $\tr_{\Y_{2}\otimes\cdots\otimes\Y_{s}}(X) = P$;
    \item[(b)] $(\I_{\X}\otimes \Pi_{\Y\ovee\Y_{2}\ovee\cdots\ovee\Y_{s}})X
                (\I_{\X}\otimes \Pi_{\Y\ovee\Y_{2}\ovee\cdots\ovee\Y_{s}}) = X$.
  \end{itemize}
\end{definition}

A further observation is that we want the semidefinite program at the first level 
of the symmetric-extension hierarchy to be at least as powerful as the program 
($\PP_{\PPT}$). In order to do so, we add a third condition to Definition 
\ref{def:bosonic}:
  \begin{itemize}
    \item[(c)] 
      \begin{description}
        \item[] $\pt_{\X}(X) \in \Pos(\X\otimes\Y\otimes\Y_{2}\otimes\cdots\otimes\Y_{s})$, and
        \item[] $\pt_{\Y_{j}\otimes\Y_{j+1}\otimes\cdots\otimes\Y_{s}}(X) \in 
      \Pos(\X\otimes\Y\otimes\Y_{2}\otimes\cdots\otimes\Y_{s})$, 
      for all $j \in \{2, \ldots, s\}$.
      \end{description}
  \end{itemize}

Finally, we can put all the constraint together in a hierarchy of semidefinite programs 
in which, at the $s$-th level, we optimize over measurements whose operators possess 
$s$-symmetric bosonic PPT extensions. 
The following is the semidefinite program corresponding to the level $s = 2$ of 
the hierarchy (the programs corresponding to higher levels of the hierarchy can 
be easily inferred from this one).

\begin{center}
\underline{Primal ($\PP_{\Sym}$)}
\begin{equation}
  \label{eq:sym-primal-problem}
  \begin{split}
    \text{maximize:} \quad & 
      \sum_{k=1}^{N} p_{k}\ip{\rho_{k}}{\mu(k)},\\
    \text{subject to:} \quad & 
      \sum_{k=1}^{N} \mu(k) = \I_{\X\otimes\Y},\\[2ex]
        & \left.\kern-\nulldelimiterspace
        \!\!\begin{aligned}
          &\tr_{\Y_{2}}(X_{k}) = \mu(k),\\[1ex]
          &(\I_{\X}\otimes \Pi_{\Y\ovee\Y_{2}})X_{k}
                  (\I_{\X}\otimes \Pi_{\Y\ovee\Y_{2}}) = X_{k},\\[1ex]
          &\pt_{\X}(X_{k}) \in \Pos(\X\otimes\Y\otimes\Y_{2}),\\[1ex]
          &\pt_{\Y_{2}}(X_{k}) \in \Pos(\X\otimes\Y\otimes\Y_{2}),
      \end{aligned}\,\right\} \quad(\text{for each}\;k = 1,\ldots,N),\\[2ex]
      \quad & X_{1}, \ldots, X_{N} \in \Pos(\X\otimes\Y\otimes\Y_{2}).
  \end{split}
\end{equation}
\end{center}
\vspace{10pt} 
The dual program can be derived by moving to the standard form and again, 
by observing that the partial transpose is its own adjoint.
\vspace{10pt} 
\begin{center}
\underline{Dual ($\DP_{\Sym}$)}
\begin{equation}
  \label{eq:sym-dual-problem}
  \begin{split}
    \text{minimize:} \quad &
      \tr(H),\\[1ex]
    \text{subject to:} 
      \quad \\[-3.5ex] 
      & \left.\kern-\nulldelimiterspace
        \!\!\begin{aligned}
          &H - Q_{k} \geq p_{k}\rho_{k},\\[1ex]
          & Q_{k}\otimes\I_{\Y_{2}} + (\I_{\X}\otimes \Pi_{\Y\ovee\Y_{2}})R_{k}(\I_{\X}\otimes \Pi_{\Y\ovee\Y_{2}}) - R_{k} \\ 
            & \qquad - \pt_{\X}(S_{k}) - \pt_{\Y}(Z_{k})
      \in \Pos(\X\otimes\Y\otimes\Y_{2}),
      \end{aligned}\,\right\} \quad(\text{for each}\;k = 1,\ldots,N),\\[2ex]
      \quad & H, Q_{1}, \ldots, Q_{N}\in\Herm(\X\otimes\Y),\\
      \quad & R_{1}, \ldots, R_{N}\in\Herm(\X\otimes\Y\otimes\Y_{2})\\
      \quad & S_{1}, \ldots, S_{N},Z_{1},\ldots,Z_{N}
        \in\Pos(\X\otimes\Y\otimes\Y_{2}).\\
  \end{split}
\end{equation}
\end{center}

\begin{figure}[!htbp]
  \centering
    \def\svgwidth{200pt}
    \scalebox{.85}{
    \input{drawing_symm.pdf_tex}}
    \caption{Symmetric extension hierarchy.}
    \label{fig:symm-extension}
\end{figure}

\begin{figure}[!htbp]
  \centering
    \def\svgwidth{200pt}
    \scalebox{.85}{
    \input{drawing_symm_dual.pdf_tex}}
    \caption{Dual of the symmetric extension hierarchy of Figure \ref{fig:symm-extension}.}
    \label{fig:symm-extension-dual}
\end{figure}

%----------------------------------------------------------------------
\section{An example: Werner hiding pair}
%----------------------------------------------------------------------

In order to demonstrate an analytic method to use the cone programming framework
presented above, we apply it here to a simple example. 
We reprove the result from Example \ref{example:werner-hiding-pairs},
that is, a tight bound on the LOCC-distinguishability of a pair of quantum hiding states.
For any positive integer $n \leq 2$, consider the equiprobable ensemble $\E^{(n)}$ of 
the two extremal Werner states
$\sigma_{0}^{(n)}, \sigma_{1}^{(n)} \in \Density(\complex^{n}\otimes\complex^{n})$
defined in Eq. \eqref{eq:werner-states}.
Here we prove that
\begin{equation}
  \opt_{\PPT}(\E^{(n)}) \leq \frac{1}{2} + \frac{1}{n+1},
\end{equation}
for any $n \geq 2$.
For this particular case, the semidefinite program ($\PP_{\PPT}$) is 
sufficient to prove a tight bound on the distinguishability of the states.
This is not always the case, as we will see in the next chapter.

Let the integer $n$ be fixed and let $\X$ and $\Y$ be copies of $\complex^{n}$.
First, we instantiate Program \eqref{eq:ppt-dual-problem} for the ensemble $\E^{(n)}$:
\begin{equation}
  \label{eq:ppt-dual-problem-werner}
  \begin{split}
    \text{minimize:} \quad & \tr(H)\\
    \text{subject to:} 
      \quad & H - \frac{1}{2}\sigma_0^{(n)} \in \PPT^{\ast}(\X:\Y),\\
      \quad & H - \frac{1}{2}\sigma_1^{(n)} \in \PPT^{\ast}(\X:\Y),\\
      \quad & H \in \Herm(\X\otimes\Y).\\
  \end{split}
\end{equation}
Next, we define the Hermitian operator $H \in \Herm(\X\otimes\Y)$ as 
\begin{equation}
\label{eq:H-werner}
  H = \frac{1}{2}\sigma_{0}^{(n)} + \frac{1}{n+1}\sigma_{1}^{(n)},
\end{equation}
and observe that the trace is equal to the bound we seek to prove, that is,
\begin{equation}
  \tr(H) = \frac{1}{2} + \frac{1}{n+1}.
\end{equation}
It is left to be checked that $H$ is a feasible solution of the dual program.
By the definition of the cone $\PPT^{\ast}(\X\otimes\Y)$, 
we need to show that there exist positive semidefinite operators 
$R_{0}, R_{1} \in \Pos(\X\otimes\Y)$ such that
\begin{equation}
H - \frac{1}{2}\sigma_0^{(n)} \geq \pt_{\X}(R_{0})
\end{equation}
and 
\begin{equation}
H - \frac{1}{2}\sigma_1^{(n)} \geq \pt_{\X}(R_{1}).
\end{equation}
Let 
\begin{equation}
u = \frac{1}{\sqrt{n}}\sum_{i=0}^{n-1}\ket{i}\ket{i}
\end{equation}
be the canonical maximally entangled state in $\X\otimes\Y$ and let 
\begin{equation}
  R_{0} = 0
    \hspace*{1cm}\mbox{and}\hspace*{1cm}
  R_{1} = \frac{1}{n+1}uu^{\ast}.
\end{equation}
We have
\begin{equation}
  H - \frac{1}{2}\sigma_{0}^{(n)} = \frac{1}{n+1}\sigma_{1}^{(n)} \geq 0 = \pt_{\X}(R_{0}),
\end{equation}
and
\begin{equation}
    H - \frac{1}{2}\sigma_{1}^{(n)} = \frac{1}{2}\sigma_{0}^{(n)} - 
      \frac{n-1}{2(n+1)}\sigma_{1}^{(n)} = \frac{1}{n(n+1)}W_{n} = \pt_{\X}(R_{1}), 
\end{equation}
where $W_{n}\in\Unitary(\X\otimes\Y)$ in the last equation is the swap operator 
defined in \eqref{eq:swap-operator}. 

%-------------------------------------------------------------------------------
\section{A discussion on computational aspects}
\label{sec:computational-aspects}
%-------------------------------------------------------------------------------

The proof approach we outlined for the Werner hiding pair (and that we are going to 
pursue all over in the following chapters) may leave an uneasy feeling to the reader, 
even though it is mathematically legitimate. We started by defining the operator $H$ in 
Eq.~\eqref{eq:H-werner} whose trace was exactly equal to the bound we wanted to prove, 
and then we proved that $H$ is indeed a feasible solution of the dual problem.
A posteriori our strategy did work out just fine, but how did we know that $H$ was 
a good candidate for the solution we were seeking?

For simple problems such as the one above, one could come up with some
insights after trying a few candidates that look promising and eventually tweak
the solution until things work out.
For more complicated instances this is not always a good strategy, and most often we are not 
blessed by magic insights. What comes to rescue in difficult situations is a numerical 
approach: we can get a good candidate solution by running an actual computer implementation of 
one of the programs described above. 
The output of the program often gives insights on a potential solution, which can 
be then verified analytically, as we did above for the Werner hiding pair example. 
When running a particular instance of a convex optimization problem, 
attention should be paid to the time and the memory space that computer solvers need,
which is the topic of discussion of the rest of this section.

Optimizing over separable operators (and therefore over separable measurements) is \mbox{NP-hard} \cite{Gharibi10}, 
which simply means that solving the cone program ($\PP_{\Sep}$) is not feasible.
However, there do exist algorithms that solve semidefinite programs efficiently.
This very fact motivated us (and \cite{Doherty04} before us) to introduce the hierarchy of semidefinite programs 
in Section \ref{sec:symm-ext}.
More precisely, let the dimensions of Alice's and Bob's subspaces be $\dim(\X) = n$ and
$\dim(\Y) = m$, respectively. Then the \emph{ellipsoid method} \cite{MartinGroetschel93} 
solves the program corresponding to the $s$-th level of the symmetric extension hierarchy
in a running time that is \emph{polynomial} in $N$ (the number of states), $n$, $m^{s}$, 
and the maximum bit-length of the entries of the density matrices specifying the 
input states\footnote{Actual software implementations of semidefinite programming
solvers may use other methods that are not guaranteed to run as efficiently in theory, 
but that behave very well in practice, the \emph{interior point method} being one of those.}.
Notice that the value of $s$ needed in order to reach a good approximation 
of the separable problem can grow larger than a constant (and in fact larger than $m^{o(1)}$), 
which is why this does not contradict the above-mentioned NP-hardness of the problem (or more precisely, 
any complexity theory hypothesis).

One way to make the algorithm more efficient in terms of memory used is to observe that 
ultimately we are really optimizing only over operators in the space
\begin{equation}
  \X\otimes\Y_{1}\ovee\cdots\ovee\Y_{s},
\end{equation}
which has dimension 
\begin{equation}
 d = n\binom{s+m-1}{m-1},
\end{equation}
that is, much smaller than $nm^{s}$. 

Let us see how this works for the case $s=2$, the other cases being
a straightforward generalization.  
Let $A \in \Unitary(\Y\otimes\Y_{2},\Y\ovee\Y_{2})$ be the linear isometry such 
that $AA^{\ast} = \Pi_{\Y\ovee\Y_{2}}$, where 
$\Pi_{\Y\ovee\Y_{2}}$ is the projection on the symmetric subspace
$\Y\ovee\Y_{2}$.
Then we can replace the two sets of constraints
\begin{equation}
\begin{aligned}
  &\tr_{\Y_{2}}(X_{k}) = \mu(k),\\
  &(\I_{\X}\otimes \Pi_{\Y\ovee\Y_{2}})X_{k}
          (\I_{\X}\otimes \Pi_{\Y\ovee\Y_{2}}) = X_{k}
\end{aligned}
\end{equation}
in the Program \eqref{eq:sym-primal-problem}, with a set of constraints of the form
\begin{equation}
  \tr_{\Y_{2}}((\I_{\X}\otimes A)X_{k}(\I_{\X}\otimes A^{\ast})) = \mu(k),
\end{equation}
where the only variables of the program are now of the form
\begin{equation}
  X_{k}\in\Pos(\complex^{d}).
\end{equation}

A MATLAB function that checks for separable/PPT distinguishability
has been developed as part of N. Johnston's QETLAB toolbox \cite{Johnston2015}. 
See Appendix~\ref{chap:AppendixA} for more details on the implementation,
as well as a tutorial on how to use the function. 

\section{Unambiguous state discrimination}

\label{sec:unambiguous-program}
In the previous sections, we analyzed the problem of distinguishing quantum states 
using measurements that minimize the probability of error.
Here we consider a strategy in which Alice and Bob can give an inconclusive, yet
never incorrect answer. Such measurement strategies are called \emph{unambiguous}.

If there are $N$ states to be distinguished, an unambiguous measurement 
\[
  \mu : \{1, \ldots, N+1\} \rightarrow \Pos(\X\otimes\Y)
\]
consists of $N+1$ operators, where the outcome of the operator $\mu(N+1)$ corresponds 
to the inconclusive answer.

The cone programming approach has already been used to study unambiguous 
discrimination by global measurement \cite{Eldar03}, 
but never, as far as we know, to study unambiguous local discrimination. 
In fact, we believe that unambiguous LOCC discrimination in general has not 
been thoroughly investigated yet.

The optimal value of the following cone program is equal to the success 
probability of unambiguously distinguishing an ensemble of states 
$\{ \rho_{1}, \ldots, \rho_{k} \}$ using measurements whose operators belong
to a convex cone $\C \subset \Pos(\X\otimes\Y)$.
Again, we assume that the states are drawn with a uniform probability.
\begin{center}
    \centerline{\underline{Primal problem}}\vspace{-4mm}
    \begin{align}
      \text{maximize:}\quad & \sum_{k = 1}^N p_{k}\ip{\mu(k)}{\rho_{k}}\notag\\
      \text{subject to:}\quad & \sum_{k=1}^{N+1} \mu(k)= \I_{\X\otimes\Y}, \label{sdp-primal-unambiguous}\\
      & \mu : \{1, \ldots, N+1\} \rightarrow \C, \notag\\
      & \ip{\mu(i)}{\rho_{j}} = 0, \qquad 1 \leq i,j \leq N, \quad i \neq j. \notag
    \end{align}
\end{center}
The dual program can be derived by routine calculation.
\begin{center}
    \centerline{\underline{Dual problem}}\vspace{-4mm}
    \begin{align}
      \text{minimize:}\quad & \tr(H)\notag\\
      \text{subject to:}\quad & H - p_{k}\rho_{k} + \sum_{\substack{1\leq i \leq N \\ i\neq k}}
          y_{i,k}\rho_{i} \in \C^{\ast}, \quad k=1,\ldots,N \; ,\label{sdp-dual-unambiguous}\\
      & H \in \Herm(\X\otimes\Y),\notag\\
      & y_{i,j} \in \real, \quad 1 \leq i,j \leq N, \quad i \neq j.\notag
    \end{align}
\end{center}
