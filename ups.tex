%!TEX root = thesis.tex
%-------------------------------------------------------------------------------
\chapter{Distinguishability of unextendable product sets}
\label{chap:ups}
%-------------------------------------------------------------------------------

In this chapter, we study the state discrimination problem for collections of
states formed by unextendable product sets.
An orthonormal collection of product vectors
\begin{equation}
  \A = \{ u_{k}\otimes v_{k} : k = 1, \ldots, N \} \subset \X \otimes \Y,
\end{equation}
for complex Euclidean spaces $\X=\complex^n$ and $\Y=\complex^m$, is said to be
an \emph{unextendable product set} if it is impossible to find a nonzero
product vector $u \otimes v \in \X \otimes \Y$ that is orthogonal to every
element of $\A$ \cite{Bennett99}.
That is, $\A$ is an unextendable product set if, for every choice of vectors
$u\in\X$ and $v\in\Y$ satisfying either $\ip{u}{u_{k}} = 0$ or 
$\ip{v}{v_{k}} = 0$ for each $k\in\{1,\ldots,N\}$, one has that either $u = 0$
or $v = 0$ (or both).

The first section establishes a simple criterion for the states formed by
any unextendable product set to be perfectly discriminated by separable
measurements, and the second subsection proves that any set of states formed
by taking the union of an unextendable product set $\A \subset \X \otimes \Y$ 
together with any pure state $z \in \X\otimes\Y$ orthogonal to every element of
$\A$ cannot be perfectly discriminated by a separable measurement.
(It is evident that PPT measurements allow a perfect discrimination in both
cases.)

\minitoc

%------------------------------------------------------------------------------%
\section{A criterion for perfect separable discrimination of
unextendable product sets}
%------------------------------------------------------------------------------%

Here we provide a simple criterion for when an unextendable product set can be
perfectly discriminated by separable measurements, and we use this criterion to
show that there is an unextendable product set $\A \subset \X\otimes\Y$ that is
not perfectly discriminated by any separable measurement when
$\X = \Y = \complex^4$.
It is known that no unextendable product set $\A \subset \X\otimes\Y$
spanning a proper subspace of $\X\otimes\Y$ can be perfectly discriminated by
an LOCC measurement \cite{Bennett99}, while every unextendable product
set can be discriminated perfectly by a PPT measurement.
It is also known that every unextendable product set $\A\subset\X\otimes\Y$ can
be perfectly discriminated by separable measurements in the case 
$\X = \Y = \complex^3$ \cite{DiVincenzo03}.

The following notation will be used throughout this section.
For $\X=\complex^n$, $\Y=\complex^m$, and
$\A = \{ u_{k}\otimes v_{k} : k = 1, \ldots, N \} \subset \X\otimes\Y$
being an unextendable product set, we will write
\begin{equation}
  \A_k = \A \backslash \{u_k \otimes v_k\},
\end{equation}
and define a set of rank-one
product projections
\begin{equation}
\label{eq:P_k-sets}
\P_k = \bigl\{ x x^{\ast} \otimes y y^{\ast}\,:\,
x\in\X,\,y\in\Y,\,\norm{x} = \norm{y} =1,\;\text{and}\;
x\otimes y\perp \A_k\bigr\}
\end{equation}
for each $k = 1,\ldots,N$.
One may interpret each element $x x^{\ast} \otimes y y^{\ast}$ of 
$\P_k$ as corresponding to a product vector $x \otimes y$ that could replace
$u_k \otimes v_k$ in $\A$, yielding a (not necessarily unextendable)
orthonormal product set.

The following theorem states that the sets $\P_1,\ldots,\P_N$ defined above
determine whether or not an unextendable product set can be perfectly
discriminated by separable measurements.

\begin{theorem}\label{thm:upb_sep_characterize}
  Let $\X=\complex^n$ and $\Y=\complex^m$ be complex Euclidean spaces and let
  \begin{equation}
    \A = \{ u_{k}\otimes v_{k} : k = 1, \ldots, N \} \subset \X\otimes\Y
  \end{equation}
  be an unextendable product set.
  The following two statements are equivalent:
  \begin{enumerate}
  \item
    There exists a separable measurement $\mu \in \Meas_{\Sep}(N, \X:\Y)$
    that perfectly discriminates the
    states represented by $\A$ (for any choice of nonzero probabilities
    $p_1,\ldots,p_N$).
  \item
    For $\P_1,\ldots,\P_N$ as defined in \eqref{eq:P_k-sets}, one has that the
    identity operator $\I_{\X}\otimes\I_{\Y}$ can be written as a nonnegative
    linear combination of projections in the set $\P_1\cup\cdots\cup \P_N$.
  \end{enumerate}
\end{theorem}

\begin{proof}
  Assume first that statement 2 holds, so that one may write
  \begin{equation}
    \I_{\X}\otimes\I_{\Y} =
    \sum_{k = 1}^N \sum_{j = 1}^{M_k} \lambda_{k,j}\,
    x_{k,j} x_{k,j}^{\ast}\otimes y_{k,j} y_{k,j}^{\ast}
  \end{equation}
  for some choice of positive integers $M_1,\ldots,M_N$, nonnegative real
  numbers $\{\lambda_{k,j}\}$, and product vectors
  $\{x_{k,j} \otimes y_{k,j}\}$ satisfying
  \begin{equation}
    x_{k,j} x_{k,j}^{\ast}\otimes y_{k,j} y_{k,j}^{\ast} \in \P_k
  \end{equation}
  for each $k\in\{1,\ldots,N\}$ and $j \in \{1,\ldots,M_k\}$.
  Define
  \begin{equation} \label{eq:P_k-enumeration}
    \mu(k) = \sum_{j = 1}^{M_k} \lambda_{k,j}\,
    x_{k,j} x_{k,j}^{\ast}\otimes y_{k,j} y_{k,j}^{\ast}
  \end{equation}
  for each $k\in\{1,\ldots,N\}$.
  It is clear that $\mu$ is a separable measurement, and by the
  definition of the sets $\P_1,\ldots,\P_N$ it necessarily holds that
  \begin{equation}
    \ip{\mu(k)}{u_\ell u_\ell^* \otimes v_\ell v_\ell^*} = 0,
  \end{equation}
  when $k\not=\ell$.
  This implies that $\mu$ perfectly discriminates the states
  represented by $\A$, and therefore implies that statement 1 holds.
  
  Now assume that statement 1 holds: there exists a separable measurement
  \[\mu \in \Meas_{\Sep}(N, \X:\Y)\] 
  that perfectly discriminates the states represented by $\A$.
  As each measurement operator $\mu(k)$ is separable, it is possible to write
  \begin{equation}
    \mu(k) = \sum_{j = 1}^{M_k} \lambda_{k,j}\,
    x_{k,j} x_{k,j}^{\ast}\otimes y_{k,j} y_{k,j}^{\ast}
  \end{equation}
  for some choice of nonnegative integers $\{M_k\}$, positive real numbers
  $\{\lambda_{k,j}\}$, and unit vectors
  $\{x_{k,j}\,:\,j=1,\ldots,M_k\}\subset\X$ and 
  $\{y_{k,j}\,:\,j=1,\ldots,M_k\}\subset\Y$.
  The assumption that this measurement perfectly discriminates $\A$ implies that
  $x_{k,j}\otimes y_{k,j} \perp \A_k$, and therefore
  $x_{k,j} x_{k,j}^{\ast} \otimes y_{k,j} y_{k,j}^{\ast} \in \P_k$, for each
  $k = 1,\ldots,N$ and $j = 1,\ldots,M_k$.
  As we have that $\mu(1)+\cdots+\mu(N) = \I_{\X} \otimes \I_{\Y}$, it follows that statement 2
  holds.
\end{proof}

It is not immediately clear that Theorem~\ref{thm:upb_sep_characterize} is
useful for determining whether or not any particular unextendable product set
can be discriminated by separable measurements, but indeed it is.
What makes this so is the fact that each set $\P_k$ is necessarily finite, as
the following proposition establishes.

\begin{prop}\label{lem:upb_finite}
  Let $\X$ and $\Y$ be complex Euclidean spaces, let
  \[
    \A = \{ u_{k}\otimes v_{k} : k = 1, \ldots, N \} \subset \X\otimes\Y
  \]
  be an unextendable product set, and let $\P_1,\ldots,\P_N$ be as defined in
  \eqref{eq:P_k-sets}.
  The sets $\P_1,\ldots,\P_N$ are finite.
\end{prop}

\begin{proof}
  Assume toward contradiction that $\P_k$ is infinite for some choice of
  $k\in\{1,\ldots,N\}$.
  There are finitely many subsets $S\subseteq \{1,\ldots,k-1,k+1,\ldots,N\}$,
  so  there must exist at least one such subset $S$ with the property that
  there are infinitely many pairwise nonparallel product vectors of the form
  $x\otimes y$ such that $x \perp u_j$ for every $j\in S$ and $y\perp v_j$ for
  every $j\not\in\S$.
  This implies that both the subspace of $\X$ orthogonal to
  $\{u_j\,:\,j\in S\}$ and the subspace of $\Y$ orthogonal to
  $\{v_j\,:\,j\not\in S\}$ have dimension at least 1, and
  at least one of them has dimension at least~2.
  It follows that there must exist a unit product vector $x \otimes y$ with
  three properties:
  (i) $x \perp u_j$ for every $j\in S$,
  (ii) $y \perp v_j$ for every $j\not\in\S$, and
  (iii) $x\otimes y\perp u_k \otimes v_k$.
  This contradicts the fact that $\A$ is unextendable, and therefore completes
  the proof.
\end{proof}

Given Proposition~\ref{lem:upb_finite}, it becomes straightforward to make use
of Theorem~\ref{thm:upb_sep_characterize} computationally.
The sets $\P_1,\ldots,\P_N$ can be computed by iterating over all 
$S \subseteq \{1,\ldots,k-1,k+1,\ldots,N\}$ and
finding the (at most one) product state orthogonal to $\{ u_j : j \in S \}$ on
$\X$ and $\{ v_j : j \notin S \}$ on $\Y$. 
Then, the second statement in Theorem~\ref{thm:upb_sep_characterize} can be
checked through the use of linear programming (and even by hand in some cases).

\begin{example}[Feng's unextendable product set]
We now present an example of an unextendable product set in $\X\otimes\Y$,
for $\X = \Y = \complex^4$, that cannot be perfectly discriminated by separable
measurements. 
In particular, let $\A$ be the unextendable product set consisting of $8$
states that were found in \cite{Feng06}:
\begin{equation}
  \begin{array}{l}
    \ket{\phi_1} = \ket{0}\ket{0}, \\
    \ket{\phi_2} = \ket{1}\left(\ket{0} - \ket{2} + \ket{3}\right)/\sqrt{3},\\
    \ket{\phi_3} = \ket{2}\left(\ket{0} + \ket{1} - \ket{3}\right)/\sqrt{3}, \\
    \ket{\phi_4} = \ket{3}\ket{3}, \\
    \ket{\phi_5} = \left(\ket{1} + \ket{2} + \ket{3}\right)\left(\ket{0}
    - \ket{1} + \ket{2}\right)/3, \\
    \ket{\phi_6} = \left(\ket{0} - \ket{2} + \ket{3}\right)\ket{2}/\sqrt{3}, \\
    \ket{\phi_7} = \left(\ket{0} + \ket{1} - \ket{3}\right)\ket{1}/\sqrt{3}, \\
    \ket{\phi_8} = \left(\ket{0} - \ket{1} + \ket{2}\right)\left(\ket{0}
    + \ket{1} + \ket{2}\right)/3.
  \end{array}
\end{equation}

For each $k = 1, \ldots, 8$, there are exactly $6$ product states contained in
$\P_k$ for each choice of $k$, which we represent by product vectors
$\ket{\phi_{k,j}}$ for $j = 1, \ldots, 6$.
To be explicit, these states are as follows (where we have omitted
normalization factors for brevity):\vspace{3mm}

\noindent\hspace{8pt}
\begin{minipage}{0.48\textwidth}
  $\ket{\phi_{1,1}} = \ket{0}\ket{0}$,\\
  $\ket{\phi_{1,2}} = \left(\ket{0} + \ket{1} - \ket{3}\right)
  \left(\ket{0} + \ket{2}\right)$,\\
  $\ket{\phi_{1,3}} = \left(\ket{0} - \ket{1}\right)
  \left(\ket{0} + \ket{1} - \ket{3}\right)$,
\end{minipage}\hfill
\begin{minipage}{0.48\textwidth}
  $\ket{\phi_{1,4}} = \left(\ket{0} - \ket{1} + 
  \ket{2}\right)\left(\ket{0} - \ket{1} + \ket{2}\right)$, \\
  $\ket{\phi_{1,5}} = \left(\ket{0} + \ket{2}\right)
  \left(\ket{0} - \ket{2} + \ket{3}\right)$,\\
  $\ket{\phi_{1,6}} = \left(\ket{0} - \ket{2} + \ket{3}\right)
  \left(\ket{0} - \ket{1}\right)$,
\end{minipage}

\vspace{2mm}

\noindent\hspace{8pt}
\begin{minipage}{0.48\textwidth}
  $\ket{\phi_{2,1}} = \ket{1}\left(\ket{0} - \ket{2} + \ket{3}\right)$,\\
  $\ket{\phi_{2,2}} = \left(\ket{0} + \ket{1} - \ket{3}\right)\ket{2}$,\\
  $\ket{\phi_{2,3}} =  \left(\ket{0} + \ket{1}\right)\ket{3}$,
\end{minipage}\hfill
\begin{minipage}{0.48\textwidth}
  $\ket{\phi_{2,4}} = \left(\ket{0} - \ket{1} + \ket{2}\right)
  \left(\ket{1} - 2\ket{2} + \ket{3}\right)$, \\
  $\ket{\phi_{2,5}} = \left(\ket{1} + \ket{2} + \ket{3}\right)
  \left(\ket{0} - \ket{1} - 2\ket{2}\right)$, \\
  $\ket{\phi_{2,6}} = \left(\ket{1} - \ket{3}\right)\ket{0}$,
\end{minipage}

\vspace{2mm}

\noindent\hspace{8pt}
\begin{minipage}{0.48\textwidth}   
  $\ket{\phi_{3,1}} = \ket{2}\left(\ket{0} + \ket{1} - \ket{3}\right)$,\\
  $\ket{\phi_{3,2}} = \left(\ket{0} - \ket{2} + \ket{3}\right)\ket{1}$,\\
  $\ket{\phi_{3,3}} =  \left(\ket{2} - \ket{3}\right)\ket{0}$,
\end{minipage}\hfill
\begin{minipage}{0.48\textwidth}
  $\ket{\phi_{3,4}} = \left(\ket{1} + \ket{2} + \ket{3}\right)
  \left(\ket{0} + 2\ket{1} + \ket{2}\right)$, \\
  $\ket{\phi_{3,5}} = \left(\ket{0} - \ket{1} + \ket{2}\right)
  \left(2\ket{1} - \ket{2} - \ket{3}\right)$, \\
  $\ket{\phi_{3,6}} = \left(\ket{0} - \ket{2}\right)\ket{3}$,
\end{minipage}

\vspace{2mm}

\noindent\hspace{8pt}
\begin{minipage}{0.48\textwidth}
  $\ket{\phi_{4,1}} = \ket{3}\ket{3}$,\\
  $\ket{\phi_{4,2}} = \left(\ket{0} + \ket{1} - \ket{2}\right)
  \left(\ket{2} + \ket{3}\right)$,\\
  $\ket{\phi_{4,3}} =  \left(\ket{1} + \ket{3}\right)
  \left(\ket{0} + \ket{1} - \ket{3}\right)$,
\end{minipage}\hfill
\begin{minipage}{0.48\textwidth}
  $\ket{\phi_{4,4}} = \left(\ket{2} + \ket{3}\right)
  \left(\ket{0} - \ket{2} + \ket{3}\right)$, \\
  $\ket{\phi_{4,5}} = \left(\ket{1} + \ket{2} + \ket{3}\right)
  \left(\ket{1} + \ket{2} + \ket{3}\right)$, \\
  $\ket{\phi_{4,6}} = \left(\ket{0} - \ket{2} + \ket{3}\right)
  \left(\ket{1} + \ket{3}\right)$,
\end{minipage}

\vspace{2mm}

\noindent\hspace{8pt}
\begin{minipage}{0.48\textwidth}
  $\ket{\phi_{5,1}} = \left(\ket{1} + \ket{2} + \ket{3}\right)
  \left(\ket{0} - \ket{1} + \ket{2}\right)$,\\
  $\ket{\phi_{5,2}} = \ket{1}\left(2\ket{0} + \ket{2} - \ket{3}\right)$,\\
  $\ket{\phi_{5,3}} = \ket{3}\ket{0}$,
\end{minipage}\hfill
\begin{minipage}{0.48\textwidth}
  $\ket{\phi_{5,4}} = \left(\ket{0} - \ket{2} - 2\ket{3}\right)\ket{2}$, \\
  $\ket{\phi_{5,5}} = \ket{2}\left(2\ket{0} - \ket{1} + \ket{3}\right)$, \\
  $\ket{\phi_{5,6}} = \left(\ket{0} + \ket{1} + 2\ket{3}\right)\ket{1}$,
\end{minipage}

\vspace{2mm}

\noindent\hspace{8pt}
\begin{minipage}{0.48\textwidth}
  $\ket{\phi_{6,1}} = \left(\ket{0} - \ket{2} + \ket{3}\right)\ket{2}$,\\
  $\ket{\phi_{6,2}} = \ket{3}\left(\ket{0} - \ket{2}\right)$,\\
  $\ket{\phi_{6,3}} =  \ket{0}\left(\ket{2} - \ket{3}\right)$,
\end{minipage}\hfill
\begin{minipage}{0.48\textwidth}
  $\ket{\phi_{6,4}} = \left(\ket{0} - \ket{1} - 2\ket{2}\right)
  \left(\ket{1} + \ket{2} + \ket{3}\right)$, \\
  $\ket{\phi_{6,5}} = \ket{2}\left(\ket{0} - \ket{2} + \ket{3}\right)$, \\
  $\ket{\phi_{6,6}} = \left(\ket{1} - 2\ket{2} + \ket{3}\right)
  \left(\ket{0} - \ket{1} + \ket{2}\right)$,
\end{minipage}

\vspace{2mm}

\noindent\hspace{8pt}
\begin{minipage}{0.48\textwidth}
  $\ket{\phi_{7,1}} = \left(\ket{0} + \ket{1} - \ket{3}\right)\ket{1}$,\\
  $\ket{\phi_{7,2}} = \ket{0}\left(\ket{1} - \ket{3}\right)$,\\
  $\ket{\phi_{7,3}} =  \ket{1}\left(\ket{0} + \ket{1} - \ket{3}\right)$,
\end{minipage}\hfill
\begin{minipage}{0.48\textwidth}
  $\ket{\phi_{7,4}} = \left(\ket{0} + 2\ket{1} + \ket{2}\right)
  \left(\ket{1} + \ket{2} + \ket{3}\right)$, \\
  $\ket{\phi_{7,5}} = \left(2\ket{1} - \ket{2} - \ket{3}\right)
  \left(\ket{0} - \ket{1} + \ket{2}\right)$, \\
  $\ket{\phi_{7,6}} = \ket{3}\left(\ket{0} + \ket{1}\right)$,
\end{minipage}

\vspace{2mm}

\noindent\hspace{8pt}
\begin{minipage}{0.48\textwidth}
  $\ket{\phi_{8,1}} = \left(\ket{0} - \ket{1} + \ket{2}\right)
  \left(\ket{0} + \ket{1} + \ket{2}\right)$,\\
  $\ket{\phi_{8,2}} = \ket{1}\left(\ket{0} - \ket{2} - 2\ket{3}\right)$,\\
  $\ket{\phi_{8,3}} =  \left(2\ket{0} - \ket{1} + \ket{3}\right)\ket{1}$,
\end{minipage}\hfill
\begin{minipage}{0.48\textwidth}
  $\ket{\phi_{8,4}} = \ket{0}\ket{3}$, \\
  $\ket{\phi_{8,5}} = \left(2\ket{0} + \ket{2} - \ket{3}\right)\ket{2}$, \\
  $\ket{\phi_{8,6}} = \ket{2}\left(\ket{0} + \ket{1} + 2\ket{3}\right)$.
\end{minipage}

\vspace{3mm}

\noindent 
One may verify by a computer that $\I\otimes\I$ is not contained in the convex
cone generated by 
\begin{equation}
  \label{eq:Feng-replacement-set}
  \bigl\{ \ket{\phi_{k,j}}\bra{\phi_{k,j}}\,:\,k = 1,\ldots,8,\;j=1,\ldots,6
  \bigr\}.
\end{equation}
(In fact, $\I\otimes\I$ is not in the linear span of the set
\eqref{eq:Feng-replacement-set}.)
Theorem~\ref{thm:upb_sep_characterize} therefore implies that this unextendable
product set is not perfectly discriminated by separable measurements.
\end{example}

The computational procedure described above was implemented in MATLAB as part 
of the QETLAB Toolbox (\cite{Johnston2015}, \texttt{UPBSepDistinguishable} function).


%------------------------------------------------------------------------------%
\section[Impossibility to distinguish an unextendable product set plus one more 
pure state]{Impossibility to distinguish an unextendable\newline
product set plus one more pure state}
%------------------------------------------------------------------------------%

Next, we prove an upper bound on the probability to correctly discriminate
any unextendable product set, together with one extra pure state orthogonal
to the members of the unextendable product set, by a separable measurement.
Central to the proof of this statement is a family of positive linear maps
previously studied in the literature \cite{Terhal01,Bandyopadhyay05}.

Before proving this fact, we note that it is fairly straightforward to obtain
a qualitative result along similar lines:
if a separable measurement were able to perfectly discriminate a particular
product set from any state orthogonal to this product set, there would
necessarily be a separable measurement operator orthogonal to the space spanned
by the product set, implying that some nonzero product state must be orthogonal
to the product set (and therefore the product set must be extendable).
Related results based on this sort of argument may be found in \cite{Bandyopadhyay11}.
An advantage of the method described here is that one obtains
precise bounds on the optimal discrimination probability, as opposed to a
statement that a perfect discrimination is not possible.

The following lemma is required for the proof of the theorem below.

\begin{lemma}[Terhal]
  \label{lemma:lambda}
  For given complex Euclidean spaces $\X=\complex^n$ and $\Y=\complex^m$, and
  any unextendable product set
  \begin{equation}
    \A = \{ u_{k}\otimes v_{k} : k = 1, \ldots, N \} \subset \X\otimes\Y,
  \end{equation}
  there exists a positive real number $\lambda_{\A} > 0$ such that
  \begin{equation}
    \left(\I_{\X} \otimes y^{\ast}\right)
    \left( \sum_{k = 1}^{N}u_{k}u_{k}^{\ast}\otimes v_{k}v_{k}^{\ast} \right)
    \left(\I_{\X} \otimes y\right)
    - \lambda_{\A}\norm{y}^{2}\I_{\X} \in \Pos(\X),
  \end{equation}
  for every $y \in \Y$.
\end{lemma}

\noindent
A proof of the lemma, as well as a constructive procedure
to calculate a bound on $\lambda_{\A}$, can be found in \cite{Terhal01}.

\begin{theorem}
  \label{thm:upb}
  Let $\X = \complex^n$ and $\Y=\complex^m$ be complex Euclidean spaces, let
  \begin{equation}
    \A = \{ u_{k}\otimes v_{k} : k = 1, \ldots, N \} \subset \X\otimes\Y
  \end{equation}
  be an unextendable product set, and let 
  \begin{equation}
    z \in \X\otimes\Y
  \end{equation}
  be a unit vector orthogonal to $\A$.
  We have that
  \begin{equation}
    \opt_{\Sep}(\A \cup \{ z \}) \leq  1 - \frac{\lambda_{\A}}{(N + 1)\delta},
  \end{equation}
  where $\lambda_{\A}$ is a positive real number satisfying the requirements 
  of Lemma~\ref{lemma:lambda} and 
  \[
    \delta = \norm{\tr_{\X}(z z^{\ast})}.
  \]
\end{theorem}

\begin{proof}
  Consider the following Hermitian operator:
  \begin{equation}
    H = \frac{1}{N+1} \left( \sum_{k = 1}^{N}u_{k}u_{k}^{\ast}\otimes 
    v_{k}v_{k}^{\ast} + \left( 1- \frac{\lambda_{\A}}{\delta} \right) 
    zz^{\ast} \right).
  \end{equation}
  We want to show that $H$ is a feasible solution of the dual problem
  \eqref{eq:sep-dual-problem} for the state discrimination problem under
  consideration.
  It is clear that
  \begin{equation}
    H - \frac{1}{N+1}u_{k}u_{k}^{\ast}\otimes v_{k}v_{k}^{\ast} \in 
    \Pos(\X\otimes\Y) \subset \BPos(\X:\Y)
  \end{equation}
  for $k = 1, \ldots, N$. 
  The remaining constraint left to be checked is the following:
  \begin{equation}
    \label{eq:upb_constraint} 
    H - \frac{1}{N+1}zz^{\ast} = 
    \frac{1}{N+1}\left(
    \sum_{k = 1}^{N}u_{k}u_{k}^{\ast}\otimes v_{k}v_{k}^{\ast} -
    \frac{\lambda_{\A}}{\delta} zz^{\ast} \right) \in \BPos(\X:\Y).
  \end{equation}
  Using the fact that
  \begin{equation}
    \delta\norm{y}^{2}\I_{\X} - 
    \left(\I_{\X} \otimes y^{\ast}\right)zz^{\ast}\left(\I_{\X} \otimes y\right)
    \in \Pos(\X),
  \end{equation} 
  for any $y \in \Y$, together with Lemma \ref{lemma:lambda}, one has that
  \begin{equation}
    \left( \I \otimes y \right)^{\ast}
    \left(\sum_{k = 1}^{N}u_{k}u_{k}^{\ast}\otimes v_{k}v_{k}^{\ast} -
    \frac{\lambda_{\A}}{\delta} zz^{\ast}\right)
    \left( \I \otimes y \right) \in \Pos(\X)
  \end{equation}
  and therefore the constraint \eqref{eq:upb_constraint} is satisfied.
  Finally, it holds that
  \begin{equation}
    \tr(H) = 1 - \frac{\lambda_{\A}}{(N + 1)\delta},
  \end{equation}
  which completes the proof.
\end{proof}

%------------------------------------------------------------------------------%
\begin{example}[Tiles set]
\label{ex:tiles-set}
%------------------------------------------------------------------------------%

Theorem \ref{thm:upb} allow us to find specific bounds for the probability 
of correctly discriminating certain sets of states.
For instance, here we consider the following unextendable product set  
$\A \subset \X\otimes\Y$ for $\X = \Y = \complex^3$, commonly known as the 
\emph{tiles set}:
\begin{equation}
  \begin{array}{llll}
    \ket{\phi_1} = \ket{0}\left(\frac{\ket{0}-\ket{1}}{\sqrt{2}}\right),
    &\ket{\phi_2} = \ket{2}\left(\frac{\ket{1}-\ket{2}}{\sqrt{2}}\right),
    &\ket{\phi_3} = \left(\frac{\ket{0}-\ket{1}}{\sqrt{2}}\right)\ket{2},
    &\ket{\phi_4} = \left(\frac{\ket{1}-\ket{2}}{\sqrt{2}}\right)\ket{0},\\[4mm]
    \multicolumn{4}{c}{\ket{\phi_5} = 
      \frac{1}{3}\left(\ket{0}+\ket{1}+\ket{2}\right)
      \left(\ket{0}+\ket{1}+\ket{2}\right).}
  \end{array}
\end{equation}
For a pure state orthogonal to this set, one may take
\begin{equation}
  \ket{\psi} = \frac{1}{2}\left(\ket{0}\ket{0} + \ket{0}\ket{1} -
  \ket{0}\ket{2} -  \ket{1}\ket{2}\right).
\end{equation}
Using the procedure described in \cite{Terhal01}, one obtains
\begin{equation}
  \lambda_{\A} \geq \frac{1}{9}\left( 1 - \sqrt{\frac{5}{6}} \right)^{2}.
\end{equation}
Therefore, if we assume that each state is selected with probability $1/6$, 
the maximum probability of correctly discriminating the set 
$\left\{ \ket{\phi_1}, \ldots, \ket{\phi_5}, \ket{\psi} \right\}$ by a
separable measurement is at most
\begin{equation}
  \opt_{\Sep}(\A) \leq 1 - \frac{1}{54}\frac{\left( 1 - \sqrt{\frac{5}{6}} \right)^{2}}
  {\cos\left(\frac{\pi}{8}\right)^{2}} <  1 - 1.647 \times 10^{-4}.
\end{equation}
This bound is not tight, as numerical optimization (see Appendix \ref{chap:AppendixA})
shows that
\begin{equation}
  \opt_{\Sep}(\A) < 0.9861 .
\end{equation} 

\end{example}

