%!TEX root = thesis.tex
%-------------------------------------------------------------------------------
\chapter{Distinguishability of maximally entangled states}
\label{chap:mes}
%-------------------------------------------------------------------------------

In this chapter we finally bring the cone programming framework into action, 
in order to answer some questions about the distinguishability of maximally entangled states.
As it was discussed in Section \ref{sec:mes-intro}, sets of maximally entangled states
constitute an important testbed for gaging the power of different classes of measurements.

As a warm-up, we first reprove a bound by \cite{Yu12} on the distinguishability
of any set of maximally entangled states.
Next, we answer an open question regarding the entanglement
cost of Bell states that was raised in \cite{Yu14}.
In the second part we study the set of states \eqref{eq:ydy_states} introduced in \cite{Yu12}. 

\minitoc

%------------------------------------------------------------------------------%
\section{General bound for maximally entangled states}
\label{sec:nathansons-bound}
%------------------------------------------------------------------------------%

We show that no PPT measurement can perfectly distinguish more than $n$ 
maximally entangled states in $\X\otimes\Y$, where $\X = \Y = \complex^{n}$. 
This result appears in \cite{Yu12} and it generalizes 
a bound by Nathanson, which is valid against LOCC and separable measurements 
\cite{Nathanson05}. The following lemma is central to the proof.
\begin{lemma}
  \label{lemma:isometry-ppt-star}
  Let $A\in\Unitary(\Y,\X)$ be a unitary operator.
  It holds that
  \begin{equation}
  \label{eq:reduction-operator}
    \I_{\X}\otimes\I_{\Y} - \vec(A) \vec(A)^{\ast} \in \PPTStar(\X:\Y).
  \end{equation}
\end{lemma}
\begin{proof}
Let $W_{n}\in\Unitary(\X,\Y)$ be the swap operator from Eq.~\eqref{eq:swap-operator}.
Consider the operator 
\begin{equation}
  U = (\overline{A}\otimes\I_{\Y})W_{n}(A^{\t}\otimes\I_{\Y}).
\end{equation}
Since $A$ and $W_{n}$ are unitary operators, so is $U$. Notice that $U$ is also Hermitian, 
and therefore its eigenvalues are either $1$ or $-1$. This implies that
\begin{equation}
  \I_{\X}\otimes\I_{\Y} -  U \in \Pos(\X\otimes\Y)
\end{equation}
is positive semidefinite. Moreover we have that
\begin{equation}
  \begin{aligned}
    \pt_{\X}(\I_{\X}\otimes\I_{\Y} -  U) &= \I_{\X}\otimes\I_{\Y} - 
      (A\otimes\I_{\Y})\vec(\I)\vec(\I)^{\ast}(A^{\ast}\otimes\I_{\Y}) \\ 
      &= \I_{\X}\otimes\I_{\Y} - \vec(A) \vec(A)^{\ast},
  \end{aligned} 
\end{equation}
and therefore 
  \begin{equation}
    \I_{\X}\otimes\I_{\Y} - \vec(A) \vec(A)^{\ast} \in \PPTStar(\X:\Y).
  \end{equation}
\end{proof}

Now, suppose that $u_1,\ldots,u_N\in\X\otimes\Y$ are vectors representing
maximally entangled pure states.
An upper-bound on the probability to distinguish these $N$ states, assuming
a uniform selection, is obtained from the dual problem \eqref{eq:ppt-dual-problem}
by considering
\begin{equation}
  H = \frac{\I_{\X}\otimes\I_{\Y}}{nN}.
\end{equation}
It holds that $H$ is a feasible solution to the dual problem:
since the states are maximally entangled, for each $k\in\{1,\ldots,N\}$ one may write
\begin{equation}
  u_k = \frac{1}{\sqrt{n}}\vec(A_k),
\end{equation}
for some choice of an isometry $A_k\in\Unitary(\Y,\X)$, and therefore
\begin{equation}
  H - \frac{1}{N}u_k u_k^{\ast} = \frac{1}{nN}\left(
  \I_{\X}\otimes\I_{\Y} - \vec(A_k) \vec(A_k)^{\ast}\right) \in \PPTStar(\X:\Y)
\end{equation}
by Lemma~\ref{lemma:isometry-ppt-star}.
Finally, the value
\begin{equation}
\tr(H) = \frac{n}{N}
\end{equation}
is an upper bound on the probability of distinguishing the states and it is smaller 
than $1$ whenever the number of states $N$ is bigger than the dimension $n$ of
each subspace.
\begin{remark}
The statement of Lemma \ref{lemma:isometry-ppt-star} may become very familiar to the reader, 
once it is translated in the language of linear mappings via the Choi isomorphism.
It is straightforward to see that the operator in Eq.~\eqref{eq:reduction-operator}
is the Choi operator of a mapping $\Phi_{A}:\Lin(\Y)\to\Lin(\X)$ defined as
\begin{equation}
  \Phi_{A}(X) = \tr(X)\I - AXA^{\ast},
\end{equation}
for any $X\in\Lin(\Y)$, which in turn is the composition of a unitary mapping
\begin{equation}
  X \to AXA^{\ast}
\end{equation}
and the mapping
\begin{equation}
  \Phi(X) = \tr(X)\I - X,
\end{equation}
which is the well-known \emph{reduction map} introduced in \cite{Horodecki99} 
(it is the mapping at the basis of the reduction criterion for entanglement detection).
In brief, we related the fact that PPT measurements can distinguish no more than 
$n$ maximally entangled states in $\complex^{n}\otimes\complex^{n}$ with the fact
that the reduction map from $\Lin(\complex^{n})\to\Lin(\complex^{n})$ is a decomposable map.
\end{remark}

%-------------------------------------------------------------------------------
\section{Entanglement cost of distinguishing Bell states}
\label{sec:entanglement-cost}
%-------------------------------------------------------------------------------

In this section, we study state discrimination problems for sets of three or
four Bell states, by LOCC, separable, and PPT measurements, with the assistance
of an entangled pair of qubits.
In particular, we will assume that Alice and Bob aim to discriminate a set of
Bell states given that they share the additional resource state
\begin{equation}
  \label{eq:tau_eps}
  \ket{\tau_{\eps}} = \sqrt{\frac{1 + \eps}{2}}\,\ket{0}\ket{0} + 
  \sqrt{\frac{1 - \eps}{2}}\,\ket{1}\ket{1},
\end{equation}
for some choice of $\eps \in [0,1]$.
The parameter $\eps$ quantifies the amount of entanglement in the state
$\ket{\tau_{\eps}}$.
Up to local unitaries, this family of states represents every pure state of two
qubits.

Using the cone programming method discussed in the previous chapter, we obtain
exact expressions for the optimal probability with which any set of three or
four Bell states can be discriminated with the assistance of the state
\eqref{eq:tau_eps} by separable measurements (which match the probabilities
obtained by LOCC measurements in all cases).

%------------------------------------------------------------------------------%
\subsection{Discriminating three Bell states}
%------------------------------------------------------------------------------%

Notice that the state $\ket{\tau_{1}} = \ket{0}\ket{0}$ is a product state and
it does not aid the two parties in discriminating any set of Bell states,
so the probability of success for $\eps = 1$ is still at most $2/3$ for a set
of three Bell states.
If $\varepsilon = 0$, then Alice and Bob can use teleportation to perfectly
discriminate all four Bell states perfectly by LOCC measurements, and therefore
the same is true for any three Bell states.
It was proved in \cite{Yu14} that PPT measurements can perfectly
discriminate any set of three Bell states using the resource state
\eqref{eq:tau_eps} if and only if $\eps \leq 1/3$.

Here we show that a maximally entangled state ($\eps = 0$) is required to 
perfectly discriminate any set of three Bell states using separable
measurements, and more generally we obtain an expression for the optimal
probability of a correct discrimination for all values of $\varepsilon$.
Because the permutations of Bell states induced by local unitaries is
transitive, there is no loss of generality in fixing the three Bell states to
be discriminated to be $\ket{\phi_1}$, $\ket{\phi_2}$, and $\ket{\phi_3}$
(as defined in \eqref{eq:Bell-states}).

\begin{theorem}
  \label{thm:three-bell}
  Let $\X_1 = \X_2 = \Y_1 = \Y_2 = \complex^2$, define 
  $\X = \X_1 \otimes \X_2$ and $\Y = \Y_1 \otimes \Y_2$, and let 
  $\eps \in [0,1]$ be chosen arbitrarily. 
  For any separable measurement $\mu\in\Meas_{\Sep}(3,\X:\Y)$, the
  success probability of correctly discriminating the states corresponding to
  the set
  \begin{equation}
    \label{eq:set-three-bells}
    \bigl\{ \ket{\phi_{1}} \otimes \ket{\tau_{\eps}},\; 
    \ket{\phi_{2}} \otimes \ket{\tau_{\eps}},\;
    \ket{\phi_{3}} \otimes \ket{\tau_{\eps}} \bigr\}
    \subset (\X_1\otimes\Y_1)\otimes(\X_2\otimes\Y_2),
  \end{equation}
  assuming a uniform distribution $p_1 = p_2 = p_3 = 1/3$, is at most
  \begin{equation}
  \label{eq:probability-three-bell}
    \frac{1}{3}\left(2 + \sqrt{1 - \eps^{2}}\right).
  \end{equation}
\end{theorem}

To prove this theorem, we require the following lemma.
The lemma introduces a family of positive maps that, to our knowledge, has not
previously appeared in the literature.

\begin{lemma}
  \label{lemma:3Bell}
  Define a linear mapping
  $\Xi_{t}: \Lin(\complex^2 \oplus \complex^2)\rightarrow
  \Lin(\complex^2 \oplus \complex^2)$ as
  \begin{equation}
    \Xi_t\begin{pmatrix}
    A & B\\
    C & D
    \end{pmatrix}
    = \begin{pmatrix}
      \Psi_t(D) + \Phi(D) &
      \Psi_t(B) + \Phi(C)\\[2mm]
      \Psi_t(C) + \Phi(B) &
      \Psi_t(A) + \Phi(A)
    \end{pmatrix}
  \end{equation}
  for every $t\in(0,\infty)$ and $A,B,C,D\in\Lin(\complex^2)$, where
  $\Psi_t:\Lin(\complex^2)\rightarrow\Lin(\complex^2)$
  is defined as
  \begin{equation}
    \Psi_t
    \begin{pmatrix}
      \alpha & \beta \\
      \gamma & \delta
    \end{pmatrix}
    = 
    \begin{pmatrix}
      t \alpha & \beta \\
      \gamma & t^{-1} \delta
    \end{pmatrix}
  \end{equation}
  and $\Phi:\Lin(\complex^2)\rightarrow\Lin(\complex^2)$ is defined as
  \begin{equation}
    \Phi\begin{pmatrix}
    \alpha & \beta \\
    \gamma & \delta
    \end{pmatrix}
    = \begin{pmatrix}
      \delta & -\beta\\
      -\gamma & \alpha
    \end{pmatrix},
  \end{equation}
  for every $\alpha,\beta,\gamma,\delta\in\complex$.
  It holds that $\Xi_t$ is a positive map for all $t\in (0,\infty)$.
\end{lemma}

\begin{proof}
  It will first be proved that $\Xi_1$ is positive.
  For every vector
  \begin{equation}
    u = \begin{pmatrix}
      \alpha\\ \beta
    \end{pmatrix}
  \end{equation}
  in $\complex^2$, define a matrix
  \begin{equation}
    M_u = \begin{pmatrix}
      \overline\alpha & \overline\beta\\[1mm]
      -\beta & \alpha
    \end{pmatrix}.
  \end{equation}
  Straightforward computations reveal that
  \begin{equation}
    M_u^{\ast} M_v = u v^{\ast} + \Phi(v u^{\ast})
    \qquad\text{and}\qquad
    M_u^{\ast} M_u = \norm{u}^2\tinyspace \I
  \end{equation}
  for all $u,v\in\complex^2$.
  It follows that
  \begin{equation}
    \Xi_1 \begin{pmatrix}
      u u^{\ast} & u v^{\ast}\\
      v u^{\ast} & v v^{\ast}
    \end{pmatrix}
    = \begin{pmatrix}
      v v^{\ast} + \Phi(v v^{\ast}) & 
      u v^{\ast} + \Phi(v u^{\ast}) \\
      v u^{\ast} + \Phi(u v^{\ast}) &
      u u^{\ast} + \Phi(u u^{\ast}) 
    \end{pmatrix}
    = \begin{pmatrix}
      \norm{v}^2 \I & M_u^{\ast} M_v\\[1mm]
      M_v^{\ast} M_u & \norm{u}^2 \I
    \end{pmatrix},
  \end{equation}
  which is positive semidefinite by virtue of the fact that
  $\norm{M_u^{\ast} M_v}\leq\norm{M_u}\norm{M_v} = \norm{u} \norm{v}$.
  As every element of $\Pos(\complex^2\oplus\complex^2)$ can be written as a
  positive linear combination of matrices of the form
  \begin{equation}
    \begin{pmatrix}
      u u^{\ast} & u v^{\ast}\\
      v u^{\ast} & v v^{\ast}
    \end{pmatrix},
  \end{equation}
  ranging over all vectors $u, v \in \complex^2$, it follows that $\Xi_1$ is a
  positive map.

  For the general case, observe first that the mapping $\Psi_s$ may be
  expressed using the Hadamard (or entry-wise) product as
  \begin{equation}
    \Psi_s
    \begin{pmatrix}
      \alpha & \beta \\
      \gamma & \delta
    \end{pmatrix}
    = 
    \begin{pmatrix}
      s \alpha & \beta \\
      \gamma & s^{-1} \delta
    \end{pmatrix}
    =
    \begin{pmatrix}
      s & 1\\
      1 & s^{-1}
    \end{pmatrix} \circ
    \begin{pmatrix}
      \alpha & \beta \\
      \gamma & \delta
    \end{pmatrix}
  \end{equation}
  for every positive real number $s\in(0,\infty)$.
  The matrix
  \begin{equation}
    \begin{pmatrix}
      s & 1\\
      1 & s^{-1}
    \end{pmatrix}
  \end{equation}
  is positive semidefinite, from which it follows (by the Schur product
  theorem) that $\Psi_s$ is a completely positive map.
  (See, for instance, Theorem 3.7 of \cite{Paulsen02}.)
  Also note that $\Phi = \Psi_s \Phi \Psi_s$ for every $s\in (0,\infty)$, which
  implies that
  \begin{equation}
    \Xi_t = \bigl(\I_{\Lin(\complex^2)} \otimes \Psi_s\bigr) \Xi_1
    \bigl(\I_{\Lin(\complex^2)} \otimes \Psi_s\bigr)
  \end{equation}
  for $s = \sqrt{t}$.
  This shows that $\Xi_t$ is a composition of positive maps for every positive
  real number~$t$, and is therefore positive.
\end{proof}

\begin{proof}[Proof of Theorem \ref{thm:three-bell}]
  For the cases that $\eps = 0$ and $\varepsilon = 1$, the theorem is known,
  as was discussed previously, so it will be assumed that $\eps \in (0,1)$.
  Define a Hermitian operator
  \begin{equation}
    H_{\eps} = \frac{1}{3}\left[\frac{\I_{\X_1\otimes\Y_1}\otimes
        \tau_{\eps}}{2} + \sqrt{1 - \eps^{2}} \, \phi_{4} \otimes
      \pt_{\negsmallspace\X_2}(\phi_{4}) \right],
  \end{equation}
  where $\tau_{\eps} = \ket{\tau_{\eps}}\bra{\tau_{\eps}}$,
  $\phi_{4} = \ket{\phi_{4}}\bra{\phi_{4}}$,
  and $\pt_{\negsmallspace\X_2}$ denotes partial transposition with respect to
  the standard basis of $\X_2$.
  It holds that
  \begin{equation}
    \tr(H_{\eps}) = \frac{1}{3}\left(2 + \sqrt{1 - \eps^{2}}\right),
  \end{equation}
  so to complete the proof it suffices to prove that $H_{\varepsilon}$ is a
  feasible solution to the dual problem \eqref{eq:sep-dual-problem}
  for the cone program corresponding to the state discrimination problem being considered.

  In order to be more precise about the task at hand, it is helpful to define a
  unitary operator $W$, mapping $\X_1\otimes\X_2\otimes\Y_1\otimes\Y_2$ to
  $\X_1\otimes\Y_1\otimes\X_2\otimes\Y_2$, that corresponds to swapping the
  second and third subsystems:
  \begin{equation}
    \label{eq:swap}
    W(x_{1}\otimes x_{2}\otimes y_{1}\otimes y_{2}) =
    x_{1}\otimes y_{1}\otimes x_{2}\otimes y_{2},
  \end{equation}
  for all vectors 
  $x_{1}\in\X_{1}$, $x_{2}\in\X_{2}$, $y_{1}\in\Y_{1}$, $y_{2}\in\Y_{2}$.
  We are concerned with the separability of measurement operators with respect
  to the bipartition between $\X_1\otimes\X_2$ and $\Y_1\otimes\Y_2$, so the
  dual feasibility of $H_{\varepsilon}$ requires that the operators defined as
  \begin{equation}
    Q_{k, \eps} = W^{\ast} \left( H_{\eps} - \frac{1}{3}\phi_{k} \otimes 
    \tau_{\eps} \right) W \in \Herm(\X\otimes\Y)
  \end{equation}
  be contained in $\BPos(\X:\Y)$ for $k = 1,2,3$.

  Let $\Lambda_{k, \eps}: \Lin(\Y) \rightarrow \Lin(\X)$ be the unique linear
  map whose Choi representation satisfies 
  $J(\Lambda_{k,\varepsilon}) = Q_{k, \eps}$ for each $k = 1,2,3$.
  As discussed in Section~\ref{sec:block-positive-operators}, 
  the block positivity of $Q_{k, \eps}$
  is equivalent to the positivity of $\Lambda_{k, \eps}$.
  Consider first the case $k = 1$ and let
  \begin{equation}
    t = \sqrt{\frac{1+\eps}{1-\eps}}.
  \end{equation}
  A calculation reveals that
  \begin{equation}
    \Lambda_{1, \eps}(Y) = \frac{\sqrt{1 - \eps^{2}}}{3}
    \left(\sigma_{3} \otimes \I_{\X_2}\right)
    \Xi_{t}(Y)
    \left(\sigma_{3} \otimes \I_{\X_2}\right),
  \end{equation}
  where $\Xi_{t}:\Lin(\Y)\rightarrow\Lin(\X)$ is the map defined in
  Lemma~\ref{lemma:3Bell} and $\sigma_{3}$ denotes one of the Pauli operators
  (see Eq.~\eqref{eq:Pauli-operators} for an explicit definition).   
  % (in general)
  % \begin{equation} \label{eq:Pauli-operators}
  % \begin{array}{llll}
  %   \sigma_{0} = \begin{pmatrix} 1 & 0 \\ 0 & 1 \end{pmatrix}, & 
  %   \sigma_{1} = \begin{pmatrix} 0 & 1 \\ 1 & 0 \end{pmatrix}, & 
  %   \sigma_{2} = \begin{pmatrix} 0 & -i \\ i & 0 \end{pmatrix}, & 
  %   \sigma_{3} = \begin{pmatrix} 1 & 0 \\ 0 & -1 \end{pmatrix}
  % \end{array}
  % \end{equation}
  % denote the Pauli operators.
  As $\eps \in (0,1)$, it holds that $t\in (0,\infty)$, and therefore
  Lemma~\ref{lemma:3Bell} implies that $\Xi_{t}(Y) \in \Pos(\X)$ for every
  $Y \in \Pos(Y)$. 
  As we are simply conjugating $\Xi_{t}(Y)$ by a unitary and scaling it 
  by a positive real factor, we also have that 
  $\Lambda_{1, \eps}(Y) \in \Pos(\X)$, for any $Y \in \Pos(Y)$, which in turn
  implies that $Q_{1, \eps} \in \BPos(\X:\Y)$.

  For the case of $k=2$ and $k = 3$, first define 
  $U, V \in \Unitary(\complex^{2})$ as follows:
  \begin{equation}
    U = \begin{pmatrix}
      1 & 0\\
      0 & i
    \end{pmatrix}
    \quad \mbox{and} \quad
    V = \frac{1}{\sqrt{2}}\begin{pmatrix}
      1 & i\\
      i & 1
    \end{pmatrix}.
  \end{equation}
  These operators transform $\phi_{1} = \ket{\phi_1}\bra{\phi_1}$ into 
  $\phi_{2} = \ket{\phi_2}\bra{\phi_2}$ and $\phi_{3} =
  \ket{\phi_3}\bra{\phi_3}$, respectively, and leave $\phi_{4}$ unchanged, in
  the following sense:
  \begin{equation}
    \begin{aligned}
      (U^{\ast}\otimes U^{\ast}) \phi_{1} (U\otimes U) = \phi_{2},\\
      (V^{\ast}\otimes V^{\ast}) \phi_{1} (V\otimes V) = \phi_{3},\\
      (U^{\ast}\otimes U^{\ast}) \phi_{4} (U\otimes U) = \phi_{4},\\
      (V^{\ast}\otimes V^{\ast}) \phi_{4} (V\otimes V) = \phi_{4}.
    \end{aligned}
  \end{equation}
  Therefore the following equations hold:
  \begin{equation}
    \begin{aligned}
      Q_{2,\eps} &= \left(U^{\ast}\otimes\I \otimes U^{\ast}\otimes\I\right) 
      Q_{1,\eps} \left(U\otimes\I \otimes U\otimes\I\right),  \\
      Q_{3,\eps} &= \left(V^{\ast}\otimes\I \otimes V^{\ast}\otimes\I\right) 
      Q_{1,\eps} \left(V\otimes\I \otimes V\otimes\I\right).
    \end{aligned}
  \end{equation}
  It follows that $Q_{2,\eps}\in\BPos(\X:\Y)$ and $Q_{3,\eps}\in\BPos(\X:\Y)$,
  which completes the proof.
\end{proof}

\begin{remark}
  The upper bound obtained in Theorem \ref{thm:three-bell} is achievable by an
  LOCC measurement, as it is the probability obtained by using the resource
  state $\ket{\tau_\varepsilon}$ to teleport the given Bell state from one
  player to the other, followed by an optimal local measurement to discriminate
  the resulting states.
\end{remark}

%------------------------------------------------------------------------------%
\subsection{Discriminating four Bell states}
%------------------------------------------------------------------------------%

It is known that, for the perfect LOCC discrimination of all four Bell states
using an auxiliary entangled state $\ket{\tau_{\varepsilon}}$ as above,
one requires that $\varepsilon = 0$ (i.e., a maximally entangled pair of qubits
is required).
This fact follows from the method of \cite{Horodecki03}, for instance.
Here we prove a more precise bound on the optimal probability of a correct
discrimination, for every choice of $\varepsilon\in[0,1]$, along similar lines
to the bound on three Bell states provided by Theorem~\ref{thm:three-bell}.
In the present case, in which all four Bell states are considered, the result
is somewhat easier: one obtains an upper bound for PPT measurements that
matches a bound that can be obtained by an LOCC measurement,
implying that LOCC, separable, and PPT measurements are equivalent for this
discrimination problem.

\begin{theorem}
  \label{thm:four-bell}
  Let $\X_1 = \X_2 = \Y_1 = \Y_2 = \complex^2$, define
  $\X = \X_1 \otimes \X_2$ and $\Y = \Y_1 \otimes \Y_2$, and let
  $\varepsilon\in [0,1]$.
  For any PPT measurement $\mu\in\Meas_{\PPT}(4,\X:\Y)$, the success
  probability of discriminating the states in the set
  \begin{equation}
    \label{eq:set-four-bells}
    \left\{ \ket{\phi_{1}} \otimes \ket{\tau_{\eps}},\; 
    \ket{\phi_{2}} \otimes \ket{\tau_{\eps}},\;
    \ket{\phi_{3}} \otimes \ket{\tau_{\eps}},\; 
    \ket{\phi_{4}} \otimes \ket{\tau_{\eps}}\right\} 
    \subset (\X_1\otimes\Y_1)\otimes(\X_2\otimes\Y_2),
  \end{equation}
  assuming a uniform distribution $p_1 = p_2 = p_3 = p_4 = 1/4$, is at most
  \begin{equation}
    \label{eq:probability-four-bell}
    \frac{1}{2}\left(1 + \sqrt{1 - \eps^2}\right).
  \end{equation}
\end{theorem}

\begin{proof}

  
  Consider the following operator:
  \begin{equation}
    H_{\varepsilon} = \frac{1}{8}
    \Bigl[
      \I_{\X_1 \otimes \Y_1} \otimes \tau_{\varepsilon}
      + \sqrt{1 - \varepsilon^2}\,
      \I_{\X_1 \otimes \Y_1} \otimes \pt_{\negsmallspace\X_2}(\phi_4)
      \Bigr] \in \Herm(\X_1\otimes\Y_1\otimes\X_2\otimes\Y_2).
  \end{equation}
  It holds that
  \begin{equation}
    \tr(H_{\eps}) = \frac{1}{2}\left( 1 + \sqrt{1-\eps^2} \right),
  \end{equation}
  so to complete the proof it suffices to prove that $H_{\varepsilon}$ is
  dual feasible for the Program \eqref{eq:ppt-dual-problem}.
  Dual feasibility will follow from the condition (which is sufficient but not necessary for feasibility)
  \begin{equation}
    (\pt_{\negsmallspace\X_1} \otimes \pt_{\negsmallspace\X_2})
    \Bigl(
    H_{\varepsilon} - \frac{1}{4}\,\phi_k\otimes\tau_{\varepsilon}\Bigr)
    \in\Pos(\X_1\otimes\Y_1\otimes\X_2\otimes\Y_2),
  \end{equation}
  for $k = 1,2,3,4$.
  One may observe that
  \begin{equation}
    \pt_{\negsmallspace\X_2}(\tau_{\varepsilon}) 
    + \frac{\sqrt{1-\varepsilon^2}}{2}\phi_4
    = \frac{1}{2}
    \begin{pmatrix}
      1 + \varepsilon & 0 & 0 & 0\\[1mm]
      0 & \frac{\sqrt{1 - \varepsilon^2}}{2} 
      & \frac{\sqrt{1 - \varepsilon^2}}{2} & 0\\[2mm]
      0 & \frac{\sqrt{1 - \varepsilon^2}}{2} 
      & \frac{\sqrt{1 - \varepsilon^2}}{2} & 0\\[2mm]
      0 & 0 & 0 & 1 - \varepsilon
    \end{pmatrix}
  \end{equation}
  is positive semidefinite, from which it follows that
  \begin{equation}
    (\pt_{\negsmallspace\X_1} \otimes \pt_{\negsmallspace\X_2})
    \Bigl(
    H_{\varepsilon} - \frac{1}{4}\,\phi_1\otimes\tau_{\varepsilon}\Bigr)
    = \frac{1}{4} \phi_4 \otimes \pt_{\negsmallspace\X_2}(\tau_{\varepsilon})
    + \frac{\sqrt{1 - \varepsilon^2}}{8} \I_{\X_1\otimes\Y_1} \otimes \phi_4
  \end{equation}
  is also positive semidefinite.
  A similar calculation holds for $k=2,3,4$, which completes the proof.
\end{proof}

\begin{remark}
  Similar to Theorem \ref{thm:three-bell}, one has that the upper bound
  obtained by Theorem \ref{thm:four-bell} is optimal for LOCC measurements,
  as it is the probability obtained using teleportation.
\end{remark}



%-------------------------------------------------------------------------------
\section{Yu-Duan-Ying states}
%-------------------------------------------------------------------------------

In this section we prove a tight bound of $3/4$ on the maximum success
probability for any LOCC measurement to discriminate the set of states 
\eqref{eq:ydy_states} exhibited by Yu, Duan, and Ying \cite{Yu12},
assuming a uniform selection of states.
The fact that this bound can be achieved by an LOCC measurement is trivial:
if Alice and Bob measure their parts of the states with respect to the standard
basis, they can easily discriminate $\ket{\phi_1}$, $\ket{\phi_2}$, and 
$\ket{\phi_4}$, erring only in the case that they receive $\ket{\phi_3}$.
The fact that this bound is optimal will be proved by exhibiting a 
feasible solution $H$ to the dual problem \eqref{eq:sep-dual-problem}
instantiated with the state discrimination problem at hand, such that
\begin{equation}
\tr(H) = \frac{3}{4}.
\end{equation}

With respect to the vector-operator correspondence, the states \eqref{eq:ydy_states} are given
by tensor products of the Pauli operators \eqref{eq:Pauli-operators} as follows:
\begin{align}
\begin{split}
  \ket{\phi_1} = \frac{1}{2}\op{vec}(U_1),\quad
   \ket{\phi_2} = \frac{1}{2}\op{vec}(U_2),\\
  \ket{\phi_3} = \frac{1}{2}\op{vec}(U_3),\quad
   \ket{\phi_4} = \frac{1}{2}\op{vec}(U_4),
\end{split}
\end{align}
for
\begin{equation}
\label{eq:ydy-operators}
\begin{aligned}
  U_1 & = 
  \sigma_0\otimes\sigma_0 = 
  \begin{pmatrix}
    1 & 0 & 0 & 0 \\
    0 & 1 & 0 & 0 \\
    0 & 0 & 1 & 0 \\
    0 & 0 & 0 & 1
  \end{pmatrix}, & 
  U_2 & = 
  \sigma_1\otimes\sigma_1 =
  \begin{pmatrix}
    0 & 0 & 0 & 1 \\
    0 & 0 & 1 & 0 \\
    0 & 1 & 0 & 0 \\
    1 & 0 & 0 & 0
  \end{pmatrix},\\[3mm]
  U_3 & = 
  i \sigma_2 \otimes \sigma_1 =
  \begin{pmatrix}
    0 & 0 & 0 & 1 \\
    0 & 0 & 1 & 0 \\
    0 & -1 & 0 & 0 \\
    -1 & 0 & 0 & 0
  \end{pmatrix}, \qquad & 
  U_4 & = 
  \sigma_3 \otimes \sigma_1 =
  \begin{pmatrix}
    0 & 1 & 0 & 0 \\
    1 & 0 & 0 & 0 \\
    0 & 0 & 0 & -1 \\
    0 & 0 & -1 & 0
  \end{pmatrix}.
\end{aligned}
\end{equation}

\subsubsection*{Bound for separable measurements}
A feasible solution of the dual problem \eqref{eq:sep-dual-problem} is based on
a construction of block positive operators that correspond, via the Choi
isomorphism, to the family of positive maps introduced by Breuer and Hall
\cite{Breuer06,Hall06}.

\begin{prop}[Breuer--Hall]
  \label{prop:breuer-hall}
  Let $\X = \Y = \complex^n$ and let $U,V\in\Unitary(\Y,\X)$ be unitary
  operators such that $U^{\t}V \in \Unitary(\Y)$ is skew-symmetric:
  $(V^{\t}U)^{\t} = -V^{\t}U$.
  It holds that
  \begin{equation}
    \I_{\X}\otimes\I_{\Y} - \vec(U) \vec(U)^{\ast} - 
        \pt_{\negsmallspace\X}(\vec(V)\vec(V)^{\ast})
    \in \BPos(\X:\Y).
  \end{equation}
\end{prop}

\begin{proof}
  For every unit vector $y\in\Y$, one has
  \begin{multline} \label{eq:BH}
    \qquad
    (\I_{\X} \otimes y^{\ast})
    (\I_{\X}\otimes\I_{\Y} - \vec(U) 
    \vec(U)^{\ast} - \pt_{\negsmallspace\X}(\vec(V)\vec(V)^{\ast}))
    (\I_{\X} \otimes y) \\
    = \I_{\X} - U \overline{y} y^{\t} U^{\ast} - \overline{V}y
    y^{\ast}V^{\t}.
    \qquad
  \end{multline}
  As it holds that $V^{\t}U$ is skew-symmetric, we have
  \begin{equation}
    \bigip{\overline{V}y}{U\overline{y}}
    = y^{\ast} V^{\t}U \overline{y} 
    = \bigip{y y^{\t}}{V^{\t}U}
    = 0,
  \end{equation}
  as the last inner product is between a symmetric and a skew-symmetric
  operator.
  Because $U$ and $V$ are unitary, it follows that
  $U\overline{y}y^{\t}U^{\ast} + \overline{V}yy^{\ast}V^{\t}$ is a rank two
  orthogonal projection, so the operator represented by \eqref{eq:BH} is also a
  projection and is therefore positive semidefinite.
\end{proof}

\begin{remark}
  The assumption of Proposition \ref{prop:breuer-hall} requires $n$ to be even,
  as skew-symmetric unitary operators exist only in even dimensions.
\end{remark}

Now, for the ensemble
\begin{equation}
\label{eq:ydy-ensemble}
  \E = \{ \ket{\phi_1}, \ket{\phi_2}, \ket{\phi_3}, \ket{\phi_4} \},
\end{equation}
one has that the following operator is a feasible solution to the dual problem
\eqref{eq:sep-dual-problem}:
\begin{equation}
  \label{eq:H_2}
  H = \frac{1}{16}(\I_{\X}\otimes\I_{\Y} - 
  \pt_{\negsmallspace\X}(\vec(V)\vec(V)^{\ast}))
\end{equation}
for
\begin{equation}
  V = i \sigma_2 \otimes \sigma_3
  = \begin{pmatrix}
    0 & 0 & 1 & 0\\
    0 & 0 & 0 & -1\\
    -1 & 0 & 0 & 0\\
    0 & 1 & 0 & 0
  \end{pmatrix}.
\end{equation}
Due to Proposition~\ref{prop:breuer-hall}, the feasibility of $H$ follows from
the condition
\begin{equation}
  (V^{\t} U_k)^{\t} = - V^{\t} U_k,
\end{equation}
which can be checked by inspecting each of the four cases.
It is easy to calculate that $\tr(H) = 3/4$, and so the required bound has been
obtained.

\subsubsection*{Bound for PPT measurements}
Interestingly, when it comes to distinguishing the Yu--Duan--Ying states, 
PPT measurements can do better than separable (and LOCC) measurement, 
yet without achieving perfect distinguishability. As far as we know, this is the
first example of a set consisting only of maximally entangled states for which 
such a gap holds.
In this section we exhibit a tight bound of $7/8$ on the probability of distinguishing
the Yu--Duan--Ying ensemble by PPT measurements.

\begin{theorem}
\label{thm:dual-ppt-ydy}
For $\E$ being the ensemble in Eq.~\eqref{eq:ydy-ensemble}, it holds that $\opt_{\PPT}(\E)\leq 7/8$.
\end{theorem}
\begin{proof}
We show that there exists a solution of the dual program $(\DP_{\PPT})$ which achieves the bound.
It is easy to check that the following operator satisfies the constraints in of the program 
and its trace is equal to $7/8$:
\[
Y = \frac{1}{16}\I\otimes\I - \frac{1}{8}\left(\psi_{2}\otimes\psi_{1}\right).
\]
We will check the constraint $Y \geq \pt_{\A}(\rho_1)$ and the reader can check the remaining constraints with
a similar calculation. By the equations in \eqref{eq:ppt-bell-states}, we have
\begin{align*}
 \pt_{\A}(\rho_{1}) &= \pt_{\A}(\psi_{0}\otimes\psi_{0}) = \left(\frac{1}{2}\I - \psi_{2}\right)\otimes\left(\frac{1}{2}\I - \psi_{2}\right) \\
&= \frac{1}{4}\I\otimes\I - \frac{1}{2}\sum_{i\in\{0,1,3\}}(\psi_{i}\otimes\psi_{2}+\psi_{2}\otimes\psi_{i}),
\end{align*}
and
\begin{align*}
 Y - \frac{1}{4}\pt_{\A}(\rho_{1}) &= \frac{1}{8}(\psi_{0}\otimes\psi_{2}+\psi_{1}\otimes\psi_{2}+\psi_{3}\otimes\psi_{2}
+\psi_{2}\otimes\psi_{0}+\psi_{2}\otimes\psi_{3}) \geq 0.
\end{align*}
\end{proof}

\begin{theorem}
\label{thm:primal-ppt-ydy}
For $\E$ being the ensemble in Eq.~\eqref{eq:ydy-ensemble}, it holds that $\opt_{\PPT}(\E)\geq 7/8$.
\end{theorem}
\begin{proof}
It is enough to show a feasible solution of the primal program ($\PP_{\PPT}$) 
that achieves the bound. 
Let $Q \in \Pos(\complex^{4}\otimes\complex^{4})$ and $R,S\in\Pos(\complex^{2}\otimes\complex^{2})$ 
be the following operators:
\[
  Q = \frac{1}{4}\I\otimes(\psi_{1}+\psi_{2}), \qquad
  R = \frac{7}{8}\psi_{0}+\frac{1}{8}\psi_{3}, \qquad 
  S = \frac{1}{8}\psi_{0}+\frac{7}{8}\psi_{3}.
\]
Then the following operators define a PPT measurement that achieves a success probability of $7/8$:
\begin{align*}
  \mu(1) &= Q + (\frac{2}{3}\psi_{0}+\frac{1}{3}\I)\otimes R ,\\
  \mu(2) &= Q + (\frac{1}{3}\psi_{0}+\psi_{1})\otimes S + \frac{1}{3}(\psi_{2}+\psi_{3})\otimes R ,\\
  \mu(3) &= Q + (\frac{1}{3}\psi_{0}+\psi_{2})\otimes S + \frac{1}{3}(\psi_{1}+\psi_{3})\otimes R ,\\
  \mu(4) &= Q + (\frac{1}{3}\psi_{0}+\psi_{3})\otimes S + \frac{1}{3}(\psi_{1}+\psi_{2})\otimes R .\\
\end{align*}
It is easy to check that these operators define a  valid measurement, that is $\sum_{k=1}^{4}\mu(k)=\I$.
Using the equations in \eqref{eq:ppt-bell-states}, we can verify that those operators are also PPT.
Again, we check this for $\mu(1)$.
\[
\pt_{\A}(\mu(1)) = 
(\psi_{1} + \psi_{2} + \psi_{4})\otimes(\frac{1}{3}\psi_{1}+\frac{1}{2}\psi_{2}+\frac{1}{3}\psi_{4})
+ \frac{1}{4}\psi_{3}\otimes(\psi_{2}+\psi_{3}) \geq 0. 
\]
Finally, we have that $\ip{\mu(k)}{\rho_{k}} = 7/8$, for each $k \in \{1,\ldots,4\}$.
\end{proof}

To recap, for $\E$ being the ensemble of Yu--Duan--Ying states selected uniformly at random, we have
\begin{equation}
  \opt_{\LOCC}(\E) = \opt_{\Sep}(\E) = \frac{3}{4} < \frac{7}{8} = \opt_{\PPT}(\E) < \opt(\E) = 1.  
\end{equation}

\subsection{Generalization to higher dimension}

Here we generalize the Yu--Duan--Ying states in order to construct sets of $N$ orthogonal 
maximally entangled states in $\complex^{N}\otimes\complex^{N}$ that are distinguishable
by separable operators only with probability $3/4$, for any $N \geq 4$ that is power of $2$. 

Let $t \geq 2$ be a positive integer and, for any choice of $k \in \{ 1, \ldots, 2^{t}\}$,
we recursively define the unitary operator
\begin{equation}
\label{eq:construction}
  U_{k}^{(t)} =
    \begin{cases}
      \sigma_{0}\otimes U_{k}^{(t-1)} &\mbox{if } 1 \leq k \leq 2^{t-1},\\
      \sigma_{1}\otimes U_{k-2^{t-1}}^{(t-1)} &\mbox{if } 2^{t-1} + 1 \leq k \leq 2^{t}.\\ 
    \end{cases}
\end{equation}
The base of the recursion, $U_{1}^{(2)}, \ldots, U_{4}^{(2)}$, 
is given by the operators defined in Eq.~\eqref{eq:ydy_Us}.
In a paper by the author \cite{Cosentino14}, it was shown that the set 
\begin{equation}
\label{eq:ydy_higher_dimension}
  \left\{ \vec(U_{k}^{(t)})\vec(U_{k}^{(t)})^{\ast} : k = 1, \ldots, 2^{t}\right\} 
\end{equation}
can be distinguished with probability at most $7/8$ by PPT measurements, 
for any value of $t \geq 2$, when the states are drawn with uniform probability, 
$p_{1}, \ldots, p_{2^{t}} = 1/2^{t}$.
In the rest of this section, we show that the maximum success probability for any separable 
measurement to distinguish the same set of states under the same assumption is $3/4$ instead.

As in the theme of this thesis, we will exhibit a feasible solution of 
the dual cone program \eqref{eq:sep-dual-problem} for which the value of 
the objective function equals $3/4$.
However, we first prove a rather general lemma, which shows how to compose a separable 
operator with a block positive operator, in order to construct another block positive 
operator in higher dimensions.
\begin{lemma}
\label{lemma:higher-dimension}
Let $\X_{1}, \X_{2}, \Y_{1}$, and $\Y_{1}$ be complex Euclidean spaces. We denote by 
$W \in \Unitary(\X_{1}\otimes\X_{2}\otimes\Y_{1}\otimes\Y_{2},
  \X_{1}\otimes\Y_{1}\otimes\X_{2}\otimes\Y_{2})$ 
the linear isometry that swaps the second and the third subsystems, which is defined by the following equation:
  \begin{equation}
  \label{eq:swap-2}
    W(x_{1}\otimes x_{2}\otimes y_{1}\otimes y_{2}) =
      x_{1}\otimes y_{1}\otimes x_{2}\otimes y_{2},
  \end{equation}
holding for all vectors $x_{1} \in \X_{1}, x_{2} \in \X_{2}, 
y_{1} \in \Y_{1}, y_{2} \in \Y_{2}$.
Let $S \in \Sep(\X_{1}:\Y_{1})$ be a separable operator 
and $Q \in \BPos(\X_{2}:\Y_{2})$ be
a block positive operator. Then the following holds:
  \begin{equation}
  W^{\ast}(S \otimes Q)W\in \BPos(\X_{1}\otimes\X_{2}:\Y_{1}\otimes\Y_{2}).
  \end{equation}
\end{lemma}
\begin{proof}
By the definition of block-positivity, the claim of the lemma is equivalent 
to the following condition:
\begin{equation}
  (\I_{\X_{1}\otimes\X_{2}} \otimes y^{\ast})W^{\ast}(S \otimes Q)W
    (\I_{\X_{1}\otimes\X_{2}} \otimes y) \in \Pos(\X_{1}\otimes\X_{2}),
\end{equation}
for every $y \in \Y_{1}\otimes\Y_{2}$.

For an arbitrary $y \in \Y_{1}\otimes\Y_{2}$, consider its Schmidt decomposition,
that is, a positive integer $r$ and orthogonal sets 
$\{w_{1}, \ldots, w_{r}\} \subset \Y_{1}$ and $\{z_{1}, \ldots, z_{r}\} \subset \Y_{2}$ 
such that
  \begin{equation}
    y = \sum_{i = 1}^{r}w_{i} \otimes z_{i}.
  \end{equation}
It holds that
\begin{multline}
\label{eq:SandQ}
    (\I\otimes y^{\ast})W^{\ast}(S \otimes Q)W(\I\otimes y)\\
    \begin{aligned}
    &=\left(\sum_{i=1}^{r}\I_{\X_{1}\otimes\X_{2}}\otimes w_{i}^{\ast}\otimes z_{i}^{\ast}\right)
      W^{\ast}(S \otimes Q)W
      \left(\sum_{i=1}^{r}\I_{\X_{1}\otimes\X_{2}}\otimes w_{i}\otimes z_{i}\right) \\
    &=\left(\sum_{i=1}^{r}\I_{\X_{1}}\otimes w_{i}^{\ast}\otimes\I_{\X_{2}} \otimes z_{i}^{\ast}\right)
    (S \otimes Q)
    \left(\sum_{i=1}^{r}\I_{\X_{1}}\otimes w_{i}\otimes\I_{\X_{2}} \otimes z_{i}\right) \\
    &= \sum_{i,j=1}^{r}(\I_{\X_{1}}\otimes w_{i}^{\ast})S(\I_{\X_{1}}\otimes w_{j})\otimes(\I_{\X_{2}} \otimes z_{i}^{\ast})Q(\I_{\X_{2}} \otimes z_{j}).
    \end{aligned}
\end{multline}
The separable operator $S \in \Sep(\X_{1}:\Y_{1})$ can be expressed as
\begin{equation}
  S = \sum_{j = 1}^{m}a_{j}a_{j}^{\ast}\otimes b_{j}b_{j}^{\ast},
\end{equation}
for vectors $a_{1}, \ldots, a_{m} \in \X_{1}$ and $b_{1}, \ldots, b_{m} \in \Y_{1}$.
By convexity, it is enough to argue about the operator
\begin{equation}
  (\I\otimes y^{\ast})W^{\ast}(aa^{\ast}\otimes bb^{\ast} \otimes Q)W(\I\otimes y), 
\end{equation}
for any vectors $a \in \X_{1}$ and $b \in \Y_{1}$. 
From Eq.~\eqref{eq:SandQ}, we have that 
\begin{multline}
  (\I\otimes y^{\ast})W^{\ast}(aa^{\ast}\otimes bb^{\ast} \otimes Q)W(\I\otimes y)\\
\begin{aligned}
  &= \sum_{i,j=1}^{r}(aa^{\ast}\otimes w_{i}^{\ast}bb^{\ast}w_{j})\otimes(\I_{\X_{2}} \otimes z_{i}^{\ast})Q(\I_{\X_{2}} \otimes z_{j})\\
  &= aa^{\ast} \otimes \left(\I_{\X_{2}}\otimes\sum_{i=1}^{r}
        w_{i}^{\ast} a z_{i}^{\ast}\right)Q
        \left(\I_{\X_{2}}\otimes\sum_{i=1}^{r} a^{\ast}w_{i}z_{i}\right) \\
  &= aa^{\ast} \otimes (\I_{\X_{2}}\otimes z^{\ast})Q
        (\I_{\X_{2}}\otimes z),
\end{aligned}
\end{multline}
where we have defined the vector $z \in \Y_{2}$ to be such that 
$z = \sum_{i=1}^{r} a^{\ast}w_{i}z_{i}$.
From the fact the $Q \in \BPos(\X_{2}:\Y_{2})$, it holds that 
\[
  (\I_{\X_{2}}\otimes z^{\ast})Q(\I_{\X_{2}}\otimes z) \in \Pos(\X_{2}),
\] 
and therefore we have that
\begin{equation}
  (\I_{\X_{1}\otimes\X_{2}} \otimes y^{\ast})W^{\ast}
    (S \otimes Q)W
    (\I_{\X_{1}\otimes\X_{2}} \otimes y) \in \Sep(\X_{1}\otimes\X_{2})
\end{equation}
is positive semidefinite.
\end{proof}

Now we are ready to show a feasible solution of the dual problem \eqref{eq:sep-dual-problem}
for the set of states \eqref{eq:ydy_higher_dimension}.
For any $t \geq 2$, we denote with $\X^{(t)}$ and $\Y^{(t)}$ two isomorphic 
copies of the complex Euclidean space $\complex^{2^{t}}$, 
so that the states in \eqref{eq:ydy_higher_dimension} lie in $\X^{(t)}\otimes\Y^{(t)}$.

Consider the operator
\begin{equation}
  S = \frac{1}{2}\vec(\sigma_{0} + \sigma_{1})\vec(\sigma_{0} + \sigma_{1})^{\ast} 
    \in \Pos(\X^{(1)}\otimes\Y^{(1)}).
\end{equation}
Let $W \in \Unitary(\X^{(t)}\otimes\Y^{(t)},\X^{(1)}\otimes\Y^{(1)}\otimes\X^{(t-1)}\otimes\Y^{(t-1)})$ 
be the linear isometry defined in the statement of Lemma \ref{lemma:higher-dimension},
acting on the specified spaces.
The solution of the dual problem may be recursively defined as follows: 
\begin{equation}
  H^{(t)} = W^{\ast}(S \otimes H^{(t-1)})W,
\end{equation}
for any $t \geq 2$. The base of the 
recursion $H^{(2)}$ is the operator defined in Eq. \eqref{eq:H_2}.
In the rest of the section we prove, by induction, that the cone program constraint
\begin{equation}
\label{eq:condiction_Ht}
  H^{(t)} - \frac{1}{2^{t}}\vec(U_{k}^{(t)})\vec(U_{k}^{(t)})^{\ast} 
      \in \BPos(\X^{(t)}:\Y^{(t)})
\end{equation}
is satisfied for any $k \in \{1, \ldots, 2^{t} \}$. 

Let us consider an arbitrary $1 \leq k \leq 2^{t-1}$ 
(the case $2^{t-1} \leq k \leq 2^{t}$ follows from a similar argument).
Eq. \eqref{eq:construction} implies that
\begin{equation}
  \begin{split}
    \vec(U_{k}^{(t)})\vec(U_{k}^{(t)})^{\ast}  
        &= W^{\ast}(\vec(\sigma_{0})\vec(\sigma_{0})^{\ast}\otimes 
          \vec(U_{k}^{(t-1)})\vec(U_{k}^{(t-1)})^{\ast})W \\
        &\leq 
         W^{\ast}(\vec(\sigma_{0} + \sigma_{1})\vec(\sigma_{0} + \sigma_{1})^{\ast} 
          \otimes \vec(U_{k}^{(t-1)})\vec(U_{k}^{(t-1)})^{\ast})W,
  \end{split} 
\end{equation}
and therefore that
\begin{align}
  H^{(t)} - \frac{1}{2^{t}}\vec(U_{k}^{(t)})\vec(U_{k}^{(t)})^{\ast} 
    &= W^{\ast}(S \otimes H^{(t-1)})W - 
      \frac{1}{2^{t}}\vec(U_{k}^{(t)})\vec(U_{k}^{(t)})^{\ast}\\
    &\geq W^{\ast}(S \otimes (H^{(t-1)} - 
          \frac{1}{2^{t-1}}\vec(U_{k}^{(t-1)})\vec(U_{k}^{(t-1)})^{\ast})W.
\end{align}
The Peres-Horodecki criterion, along with the fact that
\[
  \pt_{\X}(S) = S \in \Pos(\X_{1}\otimes\Y_{1}),
\] 
implies that $S\in\Sep(\X_{1}:\Y_{1})$. 
From Lemma \ref{lemma:higher-dimension} and the induction hypothesis, we have that
\begin{equation}
  W^{\ast}(S \otimes (H^{(t-1)} - 
      \frac{1}{2^{t-1}}\vec(U_{k}^{(t-1)})\vec(U_{k}^{(t-1)})^{\ast})W 
        \in \BPos(\X^{(t)}:\Y^{(t)}),
\end{equation}
and therefore the constraint \eqref{eq:condiction_Ht} is satisfied.
Moreover, we have that $\tr(Q) = 1$ and therefore
\[
  \tr(H^{(t)}) = \tr(H^{(2)}) = \frac{3}{4}
\]

\subsubsection*{Small sets of locally indistinguishable orthogonal maximally entangled states}

The main corollary of the proof above is that there exist 
LOCC-indistinguishable sets of $k < n$ maximally entangled states in 
$\complex^{n}\otimes\complex^{n}$.
Asymptotically, our construction allows for the cardinality $k$ of the indistinguishable 
sets to be as small as $Cn$, where $C$ is a constant less than $1$.
In particular, we have that $3/4 \leq C < 1$. 
It is possible that this constant can be improved by using 
a different construction than the Yu--Duan--Ying states.
A further improvement would be to exhibit 
indistinguishable sets of maximally entangled states with cardinality $o(n)$.
One among the smallest indistinguishable sets that come out of the above construction consists 
of the states in $\complex^{8}\otimes\complex^{8}$ corresponding the following $7$ unitary operators: 
\begin{equation}
  \label{eq:ydy_Us}
  \begin{aligned}
    U_{1} &= \sigma_{0}\otimes\sigma_{0}\otimes\sigma_{0},\\
    U_{2} &= \sigma_{0}\otimes\sigma_{1}\otimes\sigma_{1},\\
    U_{3} &= \sigma_{0}\otimes\sigma_{2}\otimes\sigma_{1},\\
    U_{4} &= \sigma_{0}\otimes\sigma_{3}\otimes\sigma_{1},\\
    U_{5} &= \sigma_{0}\otimes\sigma_{0}\otimes\sigma_{0},\\
    U_{6} &= \sigma_{0}\otimes\sigma_{1}\otimes\sigma_{1},\\
    U_{7} &= \sigma_{0}\otimes\sigma_{2}\otimes\sigma_{1}.
  \end{aligned}
\end{equation}

\begin{remark}
In a very recent result\footnote{A similar result (obtained via a different approach) appears also 
in a preprint by Yu and Oh \cite{Yu15}. Notice that this has not been
published yet, neither I have verified it myself yet.}, 
Li et al. \cite{Li15} build up on our proof from
\cite{Cosentino14} and show that indistinguishable sets of $N$ orthogonal
maximally entangled states in $\complex^{N}\otimes\complex^{N}$ exists for all 
$N$ and not just when $N$ is a power of $2$.
\end{remark}

\subsubsection*{Entanglement Discrimination Catalysis}
It is worth noting that the ``Entanglement Discrimination Catalysis'' 
phenomenon, observed in \cite{Yu12} for the set \eqref{eq:ydy_states}, 
also applies to the set 
of states in the above example and to any set derived 
from our construction.
If Alice and Bob are provided with a maximally entangled state
as a resource, then they are able to distinguish the states
in these sets and, when the protocol ends, 
they are still left with an untouched maximally entangled state.
When $t=2$, the catalyst is used to teleport 
the first qubit from one party to the other, 
say from Alice to Bob. 
Bob can then measure the first two qubits in the standard 
Bell basis and identify which of the four states was prepared. 
Since the third and fourth qubits are not being acted on, 
they can be used in a new round of the protocol.
For the case $t > 2$, let us recall the recursive construction 
of the states from \eqref{eq:construction}.
Distinguishing between the two cases of the recursion is 
equivalent to distinguishing between two Bell states.
And the base case is exactly the case $t=2$ described above, 
with only one maximally entangled state involved in the catalysis.


\subsubsection*{PPT vs. separable in the perfect discrimination of maximally entangled states}
Michael Nathanson (personal communication) raised the question whether there 
always exists a separable measurement that \emph{perfectly} distinguishes maximally entangled 
states that are known to be \emph{perfectly} distinguishable by PPT.
The construction that generalize Yu--Duan--Ying states in high dimension provides
a negative answer to Michael's question. It turns out that if we take only $7$ out of the $8$ states
coming from the construction for $t=3$ (for example, the ones corresponding to the operators in 
Eq.~\eqref{eq:ydy_Us}), they are distinguishable by separable measurement
with probability at most $6/7$, but they are perfectly distinguishable by PPT.
Unfortunately we do not have a nice-looking closed form for
the PPT measurement operators that achieve perfect distinguishability, 
but know, by running a semidefinite programming solver, that such operators exist.

\subsection{Unambiguous discrimination}

Interestingly, the optimal probability of unambiguously distinguish the set of Yu--Duan-Ying states
with PPT measurements is $3/4$, which should be compared with the success probability of $7/8$ that can be achieved with a minimum-error strategy 
(see Theorem \ref{thm:primal-ppt-ydy}). 
Using a semidefinite program solver, we were also able to verify that this bound is actually tight.

\begin{theorem}
The maximum success probability of \emph{unambiguously} 
distinguishing the ensemble in Eq.~\eqref{eq:ydy-ensemble} with PPT measurements is equal to $3/4$. 
\end{theorem}
\begin{proof}
We show a feasible solution of the dual problem \eqref{sdp-dual-unambiguous} for which the value of the objective function is $3/4$. Let
\begin{equation}
  Y = \frac{1}{16}[(\I-\psi_{1})\otimes(\I - 2\psi_{4}) + \psi_{1}\otimes(-\psi_{1}+3\psi_{2}+3\psi_{3}+\psi_{4})].
\end{equation}
and
\begin{equation}
\begin{aligned}
 Q_{1} &= [(\I - \psi_{3})\otimes\psi_{3} + \psi_{3}\otimes(\psi_{2}+\psi_{3})]/4, \\
 Q_{2} &= [(\psi_{1} + \psi_{2})\otimes\psi_{2} + \psi_{4}\otimes(\I - \psi_{2})]/4, \\
 Q_{3} &= [(\psi_{2} + \psi_{4})\otimes\psi_{2} + \psi_{1}\otimes(\I - \psi_{2})]/4, \\
 Q_{4} &= [(\psi_{1} + \psi_{4})\otimes\psi_{2} + \psi_{2}\otimes(\I - \psi_{2})]/4, \\
 Q_{5} &= (\psi_{3}\otimes\psi_{2})/4.
\end{aligned}
\end{equation}
We can use the equations in \eqref{eq:ppt-bell-states} to verify that 
the constraints of the program \eqref{sdp-dual-unambiguous} are satisfied:
\begin{equation}
Y - \frac{1}{4}\rho_{j} + \sum_{\substack{1\leq i \leq k \\ i\neq j}}\rho_{i} = \pt_{\A}(Q_{j}), \quad j=1,\ldots,4 \quad\text{and}\quad Y \geq \pt_{\A}(Q_{5}),
\end{equation}
Finally, we have $\tr(Y) = 3/4$.
\end{proof}

% move to appendix!
% \section{Generalized Bell states}

% \begin{example}[Example 1 from \cite{Bandyopadhyay11a}]
% \begin{equation}
% \ket{\phi_{00}}, \ket{\phi_{11}}, \ket{\phi_{32}}, \ket{\phi_{31}}
% \end{equation}
% \end{example}

% \begin{example}[Example 2 from \cite{Bandyopadhyay11a}]
% \begin{equation}
% \ket{\phi_{00}}, \ket{\phi_{11}}, \ket{\phi_{32}}, \ket{\phi_{31}}
% \end{equation}
% \end{example}

% \begin{example}[Example 3 from \cite{Bandyopadhyay11a}]
% \begin{equation}
% \ket{\phi_{00}}, \ket{\phi_{11}}, \ket{\phi_{32}}, \ket{\phi_{31}}
% \end{equation}
% \end{example}

