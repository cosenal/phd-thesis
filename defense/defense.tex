\documentclass{beamer}

\usepackage{array}
\usepackage{dsfont}
\usepackage[normalem]{ulem}

\makeatletter
\def\input@path{{../}}
\makeatother
\graphicspath{{../}}

\def\I{\mathds{1}}

\newcommand{\ip}[2]{\langle #1 , #2\rangle}
\newcommand{\tr}{\operatorname{Tr}}
\newcommand{\pt}{\operatorname{T}}

\def\X{\mathcal{X}}
\def\Y{\mathcal{Y}}
\def\E{\mathcal{E}}
\def\K{\mathcal{K}}
\def\W{\mathcal{W}}
\def\C{\mathcal{C}}
\def\A{\mathcal{A}}
\def\complex{\mathbb{C}}

\newcommand{\setft}[1]{\mathrm{#1}}
\newcommand{\Density}{\setft{D}}
\newcommand{\Pos}{\setft{Pos}}
\newcommand{\PPT}{\setft{PPT}}
\newcommand{\PPTStar}{\setft{PPT}^{\ast}}
\newcommand{\Unitary}{\setft{U}}
\newcommand{\Herm}{\setft{Herm}}
\newcommand{\Lin}{\setft{L}}
\newcommand{\Sep}{\setft{Sep}}
\newcommand{\BPos}{\setft{Sep}^{\ast}}
\newcommand{\LOCC}{\setft{LOCC}}
\newcommand{\Ent}{\setft{Ent}}
\newcommand{\Sym}{\setft{Sym}}
\newcommand{\Meas}{\setft{Meas}}
\newcommand{\Ens}{\setft{Ens}}

\newcommand{\microspace}{\mspace{0.5mu}}
\newcommand{\ket}[1]{
  \lvert\microspace #1 \microspace \rangle}
\newcommand{\bra}[1]{
  \langle\microspace #1 \microspace \rvert}

\def\eps{\varepsilon}

\newcommand{\reg}[1]{\mathsf{#1}}

\usepackage{colortbl}
\definecolor{Gray}{gray}{0.90}

\newtheorem{prop}[theorem]{Proposition}

\AtBeginSection[]{
  \begin{frame}
  \vfill
  \centering
  \begin{beamercolorbox}[sep=8pt,center,shadow=true,rounded=true]{title}
    \usebeamerfont{title}\insertsectionhead\par%
  \end{beamercolorbox}
  \vfill
  \end{frame}
}



\title{Quantum State Local Distinguishability via Convex Optimization}
\subtitle{PhD Defense}
\author[A. Cosentino]{Alessandro Cosentino}
\institute[UWaterloo]{University of Waterloo}
\date{August 13, 2015}

\begin{document}
  {%
    \setbeamertemplate{headline}{}
    \frame{\titlepage}
  }

  \beamertemplatenavigationsymbolsempty
  \addtobeamertemplate{navigation symbols}{}{%
    \usebeamerfont{footline}%
    \usebeamercolor[fg]{footline}%
    \hspace{1em}%
    \insertframenumber/\inserttotalframenumber
  }
  \setbeamercolor{footline}{fg=black}
  % \setbeamerfont{footline}{series=\bfseries}

  \begin{frame}
    \frametitle{Outline}
    \tableofcontents%[part=1,pausesections]
  \end{frame}

  \section{State distinguishability problem}
    
    \begin{frame}{Quantum state distinguishability problem}
        An instance of the problem is specified by an ensemble of $N$ states
        \[
            \left\{(p_{1}, \rho_{1}), \ldots, (p_{N}, \rho_{N}) \right\},
        \]
        \begin{block}{The problem}
          \begin{itemize}
          \item An index $k\in\{1, \ldots, N\}$ is picked at random with prob. $p_{k}$
          \item The state $\rho_{k}$ is prepared on a register $\reg{X}$
          \item Alice is asked to identify $k$ by means of measurements on $\reg{X}$ 
          \end{itemize}
        \end{block}
        \begin{block}{Observations}
          \begin{itemize}
          \item \emph{Classical} states are always completely distinguishable
          \item \emph{Orthogonal} quantum states are perfectly distinguishability
          \end{itemize}
        \end{block}
    \end{frame}

    \begin{frame}{Quantum state \emph{local} distinguishability problem}
        In the ensemble
        \[
          \left\{(p_{1}, \rho_{1}), \ldots, (p_{N}, \rho_{N}) \right\},
        \]
        the states $\rho_{1},\ldots,\rho_{N}\in\Density(\X\otimes\Y)$ 
        are now bipartite states.
        \begin{block}{The problem}
          \begin{itemize}
          \item An index $k\in\{1, \ldots, N\}$ is picked at random with prob. $p_{k}$
          \item The state $\rho_{k}$ is prepared on a register $\reg{X}$ shared by two parties
          \item Alice and Bob are asked to identify $k$ by Local Operations and Classical Communication (LOCC) on $\reg{X}$
          \end{itemize}
        \end{block}
        \begin{block}{Motivations}
            \begin{itemize}
                \item Understanding nonlocality/entanglement
                \item Applications to \emph{secret sharing, data hiding, channel capacity}
                \item LOCC protocols are interesting from a physical point of view
            \end{itemize}
        \end{block}
    \end{frame}

    \begin{frame}{Quantum state local distinguishability problem}
        \begin{block}{Typical assumptions}
            \begin{itemize}
                \item Sets of \emph{orthogonal} \emph{pure} states
                \item Perfect distinguishability (with probability $1$) 
                \item \emph{Uniform} probability of selection ($p_{1} = \cdots = p_{N} = 1/N$) 
            \end{itemize}
        \end{block}
        \begin{columns}[t]
            \column{.48\textwidth}
            \begin{block}{Distinguishable}
             \begin{itemize}
              \item \textit{any} 2 pure states\\
              {\footnotesize [Walgate et al., '00]}
              \item \textit{any} 3 maximally entangled states in $\complex^{3}\otimes\complex^{3}$\\
              {\footnotesize [Nathanson - '05]}
              \item \dots
             \end{itemize}
             \vspace{3pt}
             \end{block}
            \column{.48\textwidth}
            \begin{block}{Indistinguishable}
             \begin{itemize}
              \item \textit{any} set of $3$ or $4$ Bell states\\
              {\footnotesize [Ghosh et al, '01]}
              \item $9$ \emph{product} states in $\complex^{3}\otimes\complex^{3}$\\
              {\footnotesize [Bennett et al. - PRA, 1999]}
              \item \dots
              \end{itemize}
             \vspace{3pt}
             \end{block}
        \end{columns}
    \end{frame}
  
    \begin{frame}{Quantum measurements and LOCC}
        \begin{block}{$N$-outcome bipartite measurement}        
        \[
            \mu : \{ 1, \ldots, N \} \to \Pos(\X\otimes\Y)
        \]
        \[
            \sum_{k=1}^{N}\mu(k) = \I_{\X\otimes\Y}
        \]
        \end{block}
        
        \begin{block}{LOCC measurements}
          \begin{itemize}
            \item Difficult object to handle mathematically
            \item A cumbersome formal definition 
            \item Many kinds of LOCC
          \end{itemize}
          \begin{center}
            \includegraphics[scale=0.35]{locc.pdf}
            \vskip-1mm
            \hspace*\fill{\footnotesize (from Laura Man\v{c}inska's PhD thesis)}
          \end{center}
        \end{block}        
    \end{frame}

    \begin{frame}{Separable and PPT measurements}
        \begin{block}{Separable operators}
        \vspace{-15pt}
        \begin{align*}
            P = \sum_{k = 1}^{M} Q_k \otimes R_k \in \Sep(\X:\Y), \qquad M > 0,\;& Q_1,\ldots,Q_M \in \Pos(\X), \\[-10pt]  & R_1,\ldots,R_M\in\Pos(\Y)
        \end{align*}
        \end{block}
        \vspace{-10pt}
        \begin{block}{PPT (Positive-partial-transpose) operators}
        \vspace{-10pt}
        \[
            \PPT(\X:\Y) = \left\{ P \in \Pos(\X\otimes\Y) \,:\, \pt_{\X}(P) \in \Pos(\X\otimes\Y) \right\}  
        \]
        \end{block}
        \begin{block}{Separable/PPT measurements}
        \vspace{-10pt}
        \begin{equation*}
            \begin{aligned}[c]
            &\mu(k) \in \Sep(\X:\Y), \text{ or}\\
            &\mu(k) \in \PPT(\X:\Y)\\
            \end{aligned}
            \qquad\qquad
            \begin{aligned}[c]
            (\text{for each }k\in\{1,\ldots,N\})
            \end{aligned}
            \end{equation*} 
        \end{block}
    \end{frame}

    \begin{frame}{Classes of measurements}
        \begin{figure}[!ht]
            \centering
            \def\svgwidth{200pt}
            \scalebox{.60}{\input{drawing.pdf_tex}}
        \end{figure}
        \begin{itemize}
            \item A bound on PPT/Sep reflects into a bound on LOCC
            \item Optimizing over Sep is NP-hard
            \item Optimizing over PPT is computationally easy
        \end{itemize}
    \end{frame}  

  \section{Cone programming framework}

    \begin{frame}[t]{Semidefinite program for global state distinguishability}
        \begin{itemize}
          \item One of the first applications of SDP in quantum [Eldar, Megretski, Verghese, '03]
        \end{itemize}
        \vspace{10pt}
        \begin{minipage}[t]{0.48\linewidth}
            \begin{center}
            \underline{Primal (Global)}
            \begin{equation*}
              \begin{split}
                \text{max:} \quad & 
                \sum_{k=1}^{N} p_{k}\ip{\rho_{k}}{\mu(k)},\\
                \text{s.t.:} \quad & \sum_{k=1}^{N} \mu(k) = \I_{\X},\\
                  & \mu : \{1,\ldots, N\}\rightarrow \Pos(\X)
              \end{split}
            \end{equation*}
            \end{center}
        \end{minipage}\hfill
        \begin{minipage}[t]{0.48\linewidth}
            \begin{center}
            \underline{Dual (Global)}
            \vspace{10pt}
            \begin{equation*}
              \label{eq:global-dual-problem}
              \begin{split}
                \text{min:} \quad & \tr(H)\\
                \text{s.t.:} \quad & H-p_k\rho_k \in \Pos(\X)\\
                & \quad\qquad(k = 1,\ldots,N),\\
                \quad & H \in \Herm(\X).
              \end{split}
            \end{equation*}
            \end{center}
        \end{minipage}
        \vspace{10pt}
        \begin{itemize}
          \item \emph{Weak duality}: any $H$ feasible solution of the dual gives
            an upper bound on the optimal value of the primal  
        \end{itemize}
    \end{frame}

    \begin{frame}{General cone program for bipartite state distinguishability}
        \begin{minipage}[t]{0.48\linewidth}
            \begin{center}
            \underline{Primal}
            \vspace{-5pt}
            \begin{equation*}
              \begin{split}
                \text{max:} \quad & 
                \sum_{k=1}^{N} p_{k}\ip{\rho_{k}}{\mu(k)},\\
                \text{s.t.:} \quad & \sum_{k=1}^{N} \mu(k) = \I_{\X\otimes\Y},\\
                  & \boxed{\mu : \{1,\ldots, N\}\rightarrow \C}
              \end{split}
            \end{equation*}
            \end{center}
        \end{minipage}\hfill
        \begin{minipage}[t]{0.48\linewidth}
            \begin{center}
            \underline{Dual}
            \vspace{5pt}
            \begin{equation*}
              \label{eq:global-dual-problem}
              \begin{split}
                \text{min:} \quad & \tr(H)\\
                \text{s.t.:} \quad & \boxed{H-p_k\rho_k \in \C^{\ast}}\\
                & \quad\qquad(k = 1,\ldots,N),\\
                \quad & H \in \Herm(\X\otimes\Y).
              \end{split}
            \end{equation*}
            \end{center}
        \end{minipage}
        \vspace{20pt}
        \begin{itemize}
            \item $\C\subseteq\Pos(\X\otimes\Y)$ is a closed convex cone
            \item $\C^{\ast}$ is the cone dual to $\C$, that is,
            \[
              \C^{\ast} = \{Y\in\Herm(\X\otimes\Y)\,:\,\ip{X}{Y} \geq 0\;\:\text{for all $X\in\C$}\}
            \]
            \item $\Pos(\X)$ is self-dual: $\Pos(\X) = (\Pos(\X))^{\ast}$
        \end{itemize}
    \end{frame}

    \begin{frame}{\only<1>{Cone}\only<2>{\sout{Cone} Semidefinite} program for PPT distinguishability}
        \begin{minipage}[t]{0.48\linewidth}
            \begin{center}
            \underline{Primal (PPT)}
            \vspace{-5pt}
            \begin{equation*}
              \begin{split}
                \text{max:} \quad & 
                \sum_{k=1}^{N} p_{k}\ip{\rho_{k}}{\mu(k)},\\
                \text{s.t.:} \quad & \sum_{k=1}^{N} \mu(k) = \I_{\X},\\
                  & \mu : \{1,\ldots, N\}\rightarrow \PPT(\X:\Y)
              \end{split}
            \end{equation*}
            \end{center}
        \end{minipage}\hfill
        \begin{minipage}[t]{0.48\linewidth}
            \begin{center}
            \underline{Dual (PPT)}
            \vspace{5pt}
            \begin{equation*}
              \label{eq:global-dual-problem}
              \begin{split}
                \text{min:} \quad & \tr(H)\\
                \text{s.t.:} \quad & H-p_k\rho_k \in 
                    \only<1>{\PPT^{\ast}(\X:\Y)}\only<2>{\boxed{\PPT^{\ast}(\X:\Y)}}\\
                & \quad\qquad(k = 1,\ldots,N),\\
                \quad & H \in \Herm(\X).
              \end{split}
            \end{equation*}
            \end{center}
        \end{minipage}
        \vspace{5pt}
        \onslide<2>{
        \begin{block}{Decomposable operators}
        \begin{itemize}
        \item By definition,
            \vspace{-10pt}
            \[
              \PPT^{\ast}(\X:\Y) = \{ S + \pt_{\X}(R) \,:\, S, R \in \Pos(\X\otimes\Y) \}
            \]
        \item Choi-isomorphic to \emph{decomposable maps}
        \vspace{-10pt}
            \[
              \Phi(Y) = \Psi(Y) + (\pt {\circ}\, \Xi)(Y),
                \qquad\Psi, \Xi \text{ completely positive}
            \]
        \end{itemize}
        \end{block}
        }
    \end{frame}

    \begin{frame}{Cone program for separable distinguishability}
        \begin{minipage}[t]{0.48\linewidth}
            \begin{center}
            \underline{Primal (Separable)}
            \vspace{-5pt}
            \begin{equation*}
              \label{eq:sep-primal-problem}
              \begin{split}
                \text{max:} \quad & 
                  \sum_{k=1}^{N} p_{k}\ip{\rho_{k}}{\mu(k)},\\
                \text{s.t.:} \quad & \sum_{k=1}^{N} \mu(k) = \I_{\X\otimes\Y},\\
                    & \mu : \{1,\ldots, N\}\rightarrow \Sep(\X:\Y).
              \end{split}
            \end{equation*}
             \end{center}
        \end{minipage}\hfill
        \begin{minipage}[t]{0.48\linewidth}
            \begin{center}
            \underline{Dual (Separable)}
            \vspace{5pt}
            \begin{equation*}
              \label{eq:global-dual-problem}
              \begin{split}
                \text{min:} \quad & \tr(H)\\
                \text{s.t.:} \quad & H-p_k\rho_k \in 
                \only<1>{\BPos(\X:\Y)}\only<2>{\boxed{\BPos(\X:\Y)}}\\
                & \quad\qquad(k = 1,\ldots,N),\\
                \quad & H \in \Herm(\X).
              \end{split}
            \end{equation*}
            \end{center}
        \end{minipage}
        \vspace{10pt}
        \onslide<2->{
        \begin{block}{Block-positive operators}
        \begin{itemize}
            \item Choi representation of positive linear mappings:
                \[
                    \Phi(Y) \in \Pos(\X), \quad\text{for every }Y\in\Pos(\Y)
                \]
            \item $\BPos(\X:\Y) \setminus \Pos(\X\otimes\Y)$ are the 
                \emph{entanglement witnesses}
        \end{itemize}
        \end{block}
        }
    \end{frame}

    \begin{frame}{Benefits of the cone programming formulation}
      \begin{block}{Analytically}
        \begin{itemize}
          \item To prove a set is \emph{not perfectly} distinguishable, we find $H$ s.t.
            \begin{enumerate}
                \item $\tr(H) < 1$
                \item $H-p_k\rho_k \in \BPos(\X:\Y)$, for \underline{all} $k\in\{1, \ldots,N\}$
            \end{enumerate}
          \item Many families of operators in $\BPos(\X:\Y)$ are known
          \item Actual numerical bounds, as opposed to qualitative statements
        \end{itemize}
      \end{block}
      \begin{block}{Numerically}
        \begin{itemize}
          \item We can approximate the Sep cone program by a hierarchy of semidefinite programs
          \item A MATLAB implementation is now in N. Johnston's QETLAB
        \end{itemize}
      \end{block}
    \end{frame}

  \section{Applications}
  \subsection{Maximally entangled states}

    \begin{frame}{Entanglement cost of distinguishing Bell states}
        \begin{itemize}
          \item Max. prob. to LOCC-distinguish the $4$ Bell states is $1/2$
          \item Max. prob. to LOCC-distinguish any set of $3$ Bell states is $2/3$
        \end{itemize}
        \begin{block}{Entanglement cost}
            Alice and Bob are given as a resource a $2$-qubit entangled state
            \[
                \ket{\tau_{\eps}} = \sqrt{\frac{1 + \eps}{2}}\,\ket{0}\ket{0} + 
                  \sqrt{\frac{1 - \eps}{2}}\,\ket{1}\ket{1}, \qquad \eps\in[0,1]
            \]
        \end{block}
        \begin{itemize}
            \item Up to local unitaries, it characterizes all 2-qubit states
            \item $4$ Bell states: \emph{perfect} LOCC-disting. iff $\eps = 0$ [HSSH, '03] 
            \item $3$ Bell states: \emph{perfect} PPT-disting. iff $\eps \leq 1/3$ [YDY, '14]
            \item $3$ Bell states: what about separable and LOCC?   
        \end{itemize}
    \end{frame}

    \begin{frame}{Entanglement cost of distinguishing Bell states}
        \begin{theorem}[new]
        The maximum probability of distinguishing 3 Bell states is
        \[
            \frac{1}{3}\left(2 + \sqrt{1 - \eps^{2}}\right).
        \]
        \end{theorem}
        \begin{corollary}
            \begin{enumerate}
                \item Max. ent. state ($\eps=0$) is necessary for 
                    perfect discrimination
                \item Teleportation is optimal
            \end{enumerate}
        \end{corollary}
        \begin{block}{Corresponding map}
            A new positive map from $\Lin(\complex^{2}\oplus\complex^{2})$ to $\Lin(\complex^{2}\oplus\complex^{2})$
        \end{block}
    \end{frame}
 

    \begin{frame}{Local distinguishability of maximally entangled states}
        \begin{itemize}
            \item $N$ max. ent. states in $\complex^{d}\otimes\complex^{d}$ (what's the role of $d$?)
            \item A significant case because of the dual role of entanglement
        \end{itemize}
        \vspace{-10pt}
        \begin{table}[!ht]
        \centering
        \def\arraystretch{1.5}
        \begin{tabular}{|c|c|c|p{4.3cm}|}
          \hline
            & PPT & LOCC & References\\
          \hline \hline
          $N = 2, \text{any }d$     & --- & \emph{all} dist. & [Walgate et al., 2000]\\
          \hline
          $N = 3 = d$ & --- & \emph{all} dist. & [Nathanson, 2005]\\
          \hline
          \rowcolor{Gray}
          $N = 4 = d$ & \emph{some} indist. & --- & [Yu, Duan, and Ying, 2012]\\
          \hline
          $4 < N < d$ & \emph{some} indist. & --- & \emph{this thesis}\\
          \hline
          $N > d$     & \emph{all} indist. & --- & [Yu, Duan, and Ying, 2012, \newline Duan et al. 2009, \newline Ghosh et al. 2004] \\
          \hline
        \end{tabular}
        \end{table}
    \end{frame}

    \begin{frame}{Yu--Duan--Ying states}
        \begin{theorem}[YDY, '12]
        The following 4 orthogonal max. ent. states in $\complex^{4}\otimes\complex^{4}$ are
                \emph{not perfectly} distinguishable by PPT (and therefore LOCC):
        \begin{align*}
            \phi_{1} &= \psi_{0}\otimes\psi_{0} \\
            \phi_{2} &= \psi_{1}\otimes\psi_{1} \\
            \phi_{3} &= \psi_{2}\otimes\psi_{1} \\
            \phi_{4} &= \psi_{3}\otimes\psi_{1} 
        \end{align*}
        \end{theorem}
        \begin{theorem}[new]
            The optimal probability of distinguishing YDY by LOCC is $3/4$.
        \end{theorem}
        \begin{block}{Corresponding map}
            Breuer-Hall positive linear maps from $\Lin(\complex^{4})$ to $\Lin(\complex^{4})$
        \end{block}
    \end{frame}
 
 
    \begin{frame}[t]{Small sets of locally indistinguishable max. ent. states}
        \begin{table}[!ht]
        \centering
        \def\arraystretch{1.5}
        \begin{tabular}{|c|c|c|p{4.3cm}|}
          \hline
            & PPT & LOCC & References\\
          \hline \hline
          $N = 2, \text{any }d$     & --- & \emph{all} dist. & [Walgate et al., 2000]\\
          \hline
          $N = 3 = d$ & --- & \emph{all} dist. & [Nathanson, 2005]\\
          \hline 
          $N = 4 = d$ & \emph{some} indist. & --- & [Yu, Duan, and Ying, 2012]\\
          \hline
          \rowcolor{Gray}
          $4 < N < d$ & \emph{some} indist. & --- & \emph{this thesis}\\
          \hline
          $N > d$     & \emph{all} indist. & --- & [GKRS '04, DFXY '09, YDY '12] \\
          \hline
        \end{tabular}
        \end{table}
        \begin{theorem}[new]
            There exists an indistinguishable set of $N < d$ maximally entangled states in 
            $\complex^{d}\otimes\complex^{d}$
        \end{theorem}
    \end{frame}

    \begin{frame}{Small sets of locally indistinguishable max. ent. states}
        \begin{block}{Generalized Yu--Duan--Ying}
            \begin{align*}
                \phi_{1} &= \textcolor{red}{\psi_{0}} \otimes \psi_{0}\otimes\psi_{0} 
                  \qquad \phi_{5} = \textcolor{red}{\psi_{1}} \otimes\psi_{0}\otimes\psi_{0}\\
                \phi_{2} &= \textcolor{red}{\psi_{0}} \otimes \psi_{1}\otimes\psi_{1} 
                  \qquad \phi_{6} = \textcolor{red}{\psi_{1}} \otimes\psi_{1}\otimes\psi_{1}\\
                \phi_{3} &= \textcolor{red}{\psi_{0}} \otimes \psi_{2}\otimes\psi_{1} 
                  \qquad \phi_{7} = \textcolor{red}{\psi_{1}} \otimes\psi_{2}\otimes\psi_{1}\\
                \phi_{4} &= \textcolor{red}{\psi_{0}} \otimes \psi_{3}\otimes\psi_{1} 
                  \qquad \text{\sout{\ensuremath{\phi_{8} = \textcolor{red}{\psi_{1}} \otimes\psi_{3}\otimes\psi_{1}}}}
            \end{align*}
        \end{block}
        \begin{itemize}
          \item $7$ orthogonal maximally entangled states in $\complex^{8}\otimes\complex^{8}$
          \item Maximum probability for any separable measurement is $6/7$ 
          \item Perfectly distinguishable by PPT
        \end{itemize}
    \end{frame}

  \subsection{Unextendable product sets}
    \begin{frame}{Distinguishability of unextendable product sets}    
        \begin{definition}[Unextendable product sets -- UPS]
            An orthonormal collection of product vectors
            \[
              \A = \{ u_{k}\otimes v_{k} : k = 1, \ldots, N \} \subset \X \otimes \Y
            \]
            s.t. there is no (nonzero) product vector $u\otimes v \perp \A$.
        \end{definition}
        \begin{block}{Previously known}
            \begin{itemize}
                \item no UPS can be perfectly distinguished by LOCC meas.
                \item every UPS can be perfectly distinguished by PPT meas.
                \item every UPS in $\complex^{3}\otimes\complex^{3}$ can be perfectly distinguished by separable meas.
            \end{itemize}
        \end{block}
    \end{frame}
    
    \begin{frame}{UPS indistinguishable by separable measurements}
        \begin{theorem}[new]
        Feng's UPS in $\complex^{4}\otimes\complex^{4}$ is not perfectly distinguishable by Sep.
        \begin{equation*}
          \begin{array}{l}
            \ket{\phi_1} = \ket{0}\ket{0}, \\
            \ket{\phi_2} = \ket{1}\left(\ket{0} - \ket{2} + \ket{3}\right)/\sqrt{3},\\
            \ket{\phi_3} = \ket{2}\left(\ket{0} + \ket{1} - \ket{3}\right)/\sqrt{3}, \\
            \ket{\phi_4} = \ket{3}\ket{3}, \\
            \ket{\phi_5} = \left(\ket{1} + \ket{2} + \ket{3}\right)\left(\ket{0}
            - \ket{1} + \ket{2}\right)/3, \\
            \ket{\phi_6} = \left(\ket{0} - \ket{2} + \ket{3}\right)\ket{2}/\sqrt{3}, \\
            \ket{\phi_7} = \left(\ket{0} + \ket{1} - \ket{3}\right)\ket{1}/\sqrt{3}, \\
            \ket{\phi_8} = \left(\ket{0} - \ket{1} + \ket{2}\right)\left(\ket{0}
            + \ket{1} + \ket{2}\right)/3.
          \end{array}
        \end{equation*}
        \end{theorem}
        \vspace{-10pt}
        \begin{block}{Proof}
          \begin{itemize}
              \item Not the usual cone program this time... 
              \item ...but a criterion for perfect separable discrimination of UPS
                  checkable with a \emph{linear program}!
          \end{itemize}
        \end{block}       
    \end{frame}

  \section{Conclusions}
    \begin{frame}{Contributions of the thesis}
        \begin{itemize}
         \itemsep2em
            \item A framework based on cone programming to test local distinguishability 
                of quantum states
            \item Precise (often optimal) numerical bounds on the success probability of 
                distinguishing some famous sets of states
            \item A novel connection between quantum state distinguishability and the study of 
                positive linear maps
        \end{itemize}
        \begin{exampleblock}{}
          {\large\emph{``State discrimination is the most fundamental task in physics.''}}
          \vskip5mm
          \hspace*\fill{\small--- Jon Walgate (Calgary, 2005)}
        \end{exampleblock}
    \end{frame}
        
      \begin{frame}{Open questions}
        \begin{itemize}
        \itemsep1em
          \item The other way: new sets of indistinguishable states from known positive maps?
          \item What about a cone program for LOCC... 
            \begin{center}
            \underline{Dual}
              \begin{equation*}
                \label{eq:locc-dual-problem}
                \begin{split}
                  \text{min:} \quad & \tr(H)\\
                  \text{s.t.:}
                   \quad & 
                    \begin{pmatrix}
                        H - p_{1}\rho_{1} & & \\
                         & \ddots & \\
                         & & H - p_{N}\rho_{N}
                   \end{pmatrix}\in \boxed{\Meas_{\LOCC}^{\ast}(N, \X:\Y)},\\
                \end{split}
              \end{equation*}
            \end{center}
            Can we say anything about $\Meas_{\LOCC}^{\ast}(N, \X:\Y)$?
          \item Other settings (multipartite, two copies, etc.)
        \end{itemize}
    \end{frame}

    \begin{frame}{Acknowledgments}
        \begin{center}
            \newcolumntype{L}{>{\scshape}l}
            \begin{tabular}{ r L }
                Supervisor & John Watrous \\[1em]
                Committee members & Richard Cleve \\
                                  & Runyao Duan \\
                                  & Debbie Leung \\
                                  & Ashwin Nayak \\[1em]
                Co-authors & Som Bandyopadhyay \\
                           & Nathaniel Johnston \\
                           & Vincent Russo \\
                           & Nengkun Yu \\[1em]
                Funded by  & NSERC \\
                           & US Army Research Office \\
                           & David R. Cheriton Scholarship
            \end{tabular}
        \end{center}
    \end{frame}

    \begin{frame}{Publications}
    
        \begin{itemize}
        \setlength\itemsep{1em}
            \item 
            A. Cosentino. 

            \textbf{PPT-indistinguishable states via semidefinite programming}. 
            
            \textit{Physical Review A}, 2013.

            \item
            A. Cosentino and V. Russo.

            \textbf{Small sets of locally indistinguishable orthogonal maximally entangled states}. 
            
            \textit{Quantum Information \& Computation},  2014.

            \item 
            S. Bandyopadhyay, A. Cosentino, N. Johnston, V. Russo, J. Watrous, and N. Yu. 
            
            \textbf{Limitations on separable measurements by convex optimization}. 
            
            \textit{IEEE Transactions on Information Theory}, 2015.
        \end{itemize}

    \end{frame}

\end{document}
