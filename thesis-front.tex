%!TEX root = thesis.tex
%
% T I T L E   P A G E
% -------------------
% Last updated May 24, 2011, by Stephen Carr, IST-Client Services
% The title page is counted as page `i' but we need to suppress the
% page number.  We also don't want any headers or footers.
\pagestyle{empty}
\pagenumbering{roman}

% The contents of the title page are specified in the "titlepage"
% environment.
\begin{titlepage}
        \begin{center}
        \vspace*{1.0cm}

        \Huge
        {\bf Quantum State Local Distinguishability via Convex Optimization}

        \vspace*{1.0cm}

        \normalsize
        by \\

        \vspace*{1.0cm}

        \Large
        Alessandro Cosentino \\

        \vspace*{3.0cm}

        \normalsize
        A thesis \\
        presented to the University of Waterloo \\ 
        in fulfillment of the \\
        thesis requirement for the degree of \\
        Doctor of Philosophy \\
        in \\
        Computer Science \\

        \vspace*{2.0cm}

        Waterloo, Ontario, Canada, 2015 \\

        \vspace*{1.0cm}

        \end{center}
\end{titlepage}

\noindent\textbf{Copyright notice.}
Chapters~\ref{chap:programs} and~\ref{chap:mes} contain material 
from~\cite{Cosentino13}, which is copyrighted by the American Physical Society.
Chapters~\ref{chap:programs},~\ref{chap:mes} and ~\ref{chap:ups}
contain material from~\cite{Cosentino14}, which is copyrighted by Rinton Press, 
and from~\cite{Bandyopadhyay15}, copyrighted by IEEE.
\\ \\
Remaining material is:
\copyright\ Alessandro Cosentino 2015 \\


% The rest of the front pages should contain no headers and be numbered using Roman numerals starting with `ii'
\pagestyle{plain}
\setcounter{page}{2}

\cleardoublepage % Ends the current page and causes all figures and tables that have so far appeared in the input to be printed.
% In a two-sided printing style, it also makes the next page a right-hand (odd-numbered) page, producing a blank page if necessary.
 


% D E C L A R A T I O N   P A G E
% -------------------------------
  % The following is the sample Delaration Page as provided by the GSO
  % December 13th, 2006.  It is designed for an electronic thesis.
  \noindent
I hereby declare that I am the sole author of this thesis. This is a true copy of the thesis, including any required final revisions, as accepted by my examiners.

  \bigskip
  
  \noindent
I understand that my thesis may be made electronically available to the public.

\cleardoublepage
%\newpage

% A B S T R A C T
% ---------------

\begin{center}\textbf{Abstract}\end{center}

Entanglement and nonlocality play a fundamental role in quantum computing.
To understand the interplay between these phenomena, researchers have considered the 
model of local operations and classical communication, or LOCC for short,
which is a restricted subset of all possible operations 
that can be performed on a multipartite quantum system.
The task of distinguishing states from a set that is known a priori to all parties
is one of the most basic problems among those used to test the power of LOCC protocols,
and it has direct applications to quantum data hiding, secret sharing and quantum channel capacity.

The focus of this thesis is on state distinguishability problems for 
classes of quantum operations that are more powerful than LOCC, 
yet more restricted than global operations, namely the 
classes of separable and positive-partial-transpose (PPT) measurements. 
We build a framework based on convex optimization (on cone programming, in particular) 
to study such problems.
Compared to previous approaches to the problem,
the method described in this thesis provides precise numerical bounds and quantitative analytic results.
By combining the duality theory of cone programming with the channel-state duality, 
we also establish a novel connection between the state distinguishability problem 
and the study of positive linear maps, which is a topic of independent interest in quantum information theory.

We apply our framework to several questions that were left open in previous works 
regarding the distinguishability  of maximally entangled states and unextendable product sets. 
First, we exhibit sets of $k < n$ orthogonal maximally entangled states in $\complex^{n}\otimes\complex^{n}$ 
that are not perfectly distinguishable by LOCC.
As a consequence of this, we show a gap between the power of PPT and separable measurements 
for the task of distinguishing sets consisting only of maximally entangled states.
Furthermore, we prove tight bounds on the entanglement cost that is necessary to
distinguish any sets of Bell states, thus showing that quantum teleportation is optimal for this task.
Finally, we provide an easily checkable characterization of when an unextendable product set is 
perfectly discriminated by separable measurements, along with the first known 
example of an unextendable product set that cannot be perfectly discriminated 
by separable measurements.

\cleardoublepage
\newpage

% A C K N O W L E D G E M E N T S
% -------------------------------

\begin{center}\textbf{Acknowledgements}\end{center}

First and foremost, I would like to thank my advisor John Watrous for his guidance
over my PhD years.
His meticulous style of writing has been and always will be an inspiration for me.  
I also thank him for teaching two terrific courses on quantum information theory 
and semidefinite programming. From his lectures I learned almost all the mathematical tools
I needed for writing this thesis.

Next, I am grateful to professors Richard Cleve, Runyao Duan, Debbie Leung, and Ashwin Nayak 
for agreeing to be in my defense committee, and to professor Andrew Childs for being in
my comprehensive exam committee.

Most of the questions addressed in this thesis were explained to me by Som Bandyopadhyay, 
Nengkun Yu, and Michael Nathanson. I thank them for a lot of insightful discussions on the problem.
The ``state distinguishability'' research community is small, but has welcomed me warmly.

I owe some of the ideas presented in this work to my colleagues and co-authors 
Vincent Russo and Nathaniel Johnston. I must thank them also for helping me
with coding in MATLAB the optimization problems presented in this thesis.

From my great nerdy friends Robin and Vinayak, I learned too much: lots of theoretical computer science, 
many research skills and, more importantly, rationality.

I wish to thank grad school fellows Stacey, Adam, Ansis, Jamie, Laura, and Abel, for making
my life at IQC more enjoyable, and for being the coolest bunch of quantum kids in the world.
I hope our paths will cross many times in the years to come. 

Thanks to my roommate Daniele for bringing a piece of Italy into my everyday life. 
(And I am not talking \emph{only} about Baricelle olive oil!)

To all my soccer team-mates, I'm very sad 
that I will no longer be playing for Hopeless Experts.
Every game, every defeat, every win with you guys has been fantastic. 
I wish the team every success in the future seasons.  

Throughout the past years, my \emph{classical} friend Diego has been reminding me of 
the joy of computer hacking and of how damn interesting every branch of computer science can be.
Thanks for that and for keeping me updated on your favorite musical discoveries.

I have to save my last thanks to my family for their love and support. 
I missed you!


\cleardoublepage
\newpage

% D E D I C A T I O N
% -------------------

% \begin{center}\textbf{Dedication}\end{center}

% This is dedicated to the one I love.
% \cleardoublepage
%\newpage

% T A B L E   O F   C O N T E N T S
% ---------------------------------
\renewcommand\contentsname{Table of Contents}
\tableofcontents
\cleardoublepage
\phantomsection
%\newpage

% L I S T   O F   T A B L E S
% ---------------------------
\addcontentsline{toc}{chapter}{List of Tables}
\listoftables
\cleardoublepage
\phantomsection		% allows hyperref to link to the correct page
%\newpage

% L I S T   O F   F I G U R E S
% -----------------------------
\addcontentsline{toc}{chapter}{List of Figures}
\listoffigures
\cleardoublepage
\phantomsection		% allows hyperref to link to the correct page
%\newpage

% L I S T   O F   S Y M B O L S
% -----------------------------
% To include a Nomenclature section
% \addcontentsline{toc}{chapter}{\textbf{Nomenclature}}
% \renewcommand{\nomname}{Nomenclature}
% \printglossary
% \cleardoublepage
% \phantomsection % allows hyperref to link to the correct page
% \newpage

% Change page numbering back to Arabic numerals
\pagenumbering{arabic}

