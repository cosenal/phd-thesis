%!TEX root = thesis.tex
%-------------------------------------------------------------------------------
\chapter{Introduction}
\label{chap:introduction}
%-------------------------------------------------------------------------------


%-------------------------------------------------------------------------------
\section{Motivation}
%-------------------------------------------------------------------------------

A central subject of study in quantum information theory is 
the interplay between entanglement and nonlocality.
An important tool to study this relationship is the paradigm
of local quantum operations and classical communication (LOCC). 
This is a subset of all global quantum operations, 
with a fairly intuitive physical description. 
In a two-party LOCC protocol, 
Alice and Bob can perform quantum operations only on their local
subsystems and the communication must be classical.
This restricted paradigm has played a crucial role in the understanding 
of the role of entanglement in quantum information. It has also provided a 
framework for the description of basic quantum tasks such as quantum key 
distribution and entanglement distillation.
Furthermore, LOCC protocols are operationally well-motivated, in the sense
that classical communication is easy to implement.

The LOCC paradigm does not have a proper classical counterpart.
It is worth noticing that its definition does not impose any restriction on the amount of 
classical communication that is allowed between the parties, and therefore it should not be
confused with other setups studied in theoretical computer science where such 
constraints are instead imposed, such as communication complexity, or information complexity.  

A fundamental problem that has been studied to understand the limitations of 
LOCC protocols is the problem of distinguishing quantum states.
The setup of the problem is pretty simple in the bipartite case.
The two parties are given a single copy of a quantum state chosen
with some probability from a collection of states and their goal is to identify 
which state was given, with the assumption that the parties have full 
a priori knowledge of the collection.

When restricted to classical states, this is an easy task, 
different strings of bits are always completely distinguishable. 
In the quantum case, if the states are orthogonal and global operations 
are permitted, then it is always possible to determine the state
with certainty.
When the states are not orthogonal, quantum mechanics does not allow 
perfect discrimination \cite{Nielsen11}.
The problem of distinguishing quantum states by global operations 
dates back to the '70s \cite{Helstrom1969},
and since then it has been given different names: \emph{quantum state discrimination}, 
\emph{quantum detection}, \emph{quantum hypothesis testing}.


Even when the states are orthogonal, things get interesting in the quantum setting 
once we impose 
restrictions on the measurements that can be performed on the unknown state.
Say the two parties to whom the state is given, Alice and Bob,
have their quantum labs very far apart from each other's and, say, their research budget 
pays only for an infrastructure to communicate with each other on a classical network.
In other words, say that only LOCC measurements are allowed on the state. 
Then Alice and Bob cannot in general discover the state they have been given, 
even if the states are orthogonal.

The problem of distinguishing among a known set 
of orthogonal quantum states by LOCC protocols has been studied by several researchers\footnote{The reader may want to browse through the list of references of this thesis, for a more complete list.}
\cite{Bennett99,Walgate00,Ghosh01,Horodecki03,Fan04,Ghosh04,Nathanson05,Watrous05,Yu11,Yu12}.
It is referred to as the \emph{local state distinguishability problem} (or 
\emph{local state discrimination}) and it has some direct applications 
to other problems in quantum information theory, 
such as secret sharing \cite{Cleve99,Gottesman00}, 
data hiding \cite{Terhal01a,DiVincenzo2002}, and the study of quantum channel 
capacity (see \cite{Watrous05,Yu11} and references therein).

Local state distinguishability problems offer insights into how useful entanglement 
is in quantum information processing tasks.
The reason why investigating these problems is helpful comes from the fact that the role of entanglement 
in such tasks is twofold. On the one hand, many LOCC protocols, such as the ones based on teleportation, 
are fueled by entanglement shared by the parties, and therefore entanglement is a helpful resource
for distinguishability. 
On the other hand, if the states to be distinguished are themselves entangled,
local measurements on only a part of the states do not always suffice to reveal 
all the information hidden in the remaining part.
This observation has led us towards the choice of the sets to analyze in this work, 
which ended up being of two antipodal categories: sets consisting only of orthogonal maximally entangled states
and sets consisting only of product states.

The set of measurements that can be implemented through LOCC has an apparently
complex mathematical structure---no tractable characterization of this set is
known, representing a clear obstacle to a better understanding of the
limitations of LOCC measurements.
For example, given a collection of operators 
describing a measurement on a bipartite system, the determination of whether or
not this collection describes an LOCC measurement, or is closely approximated
by an LOCC measurement, is not known to be a computationally decidable
problem.
For this reason, the state discrimination problem described above is sometimes
considered for more tractable classes of measurements that approximate, in some
sense, the set of LOCC measurements, and that are mathematically and computationally 
more tractable than the LOCC set.
Among these classes, the set of separable and positive-partial-transpose (PPT) measurements 
are the subject of study of this thesis.
Since these classes contain LOCC, any bound on their power is reflected into a 
bound on the power of LOCC.

The key observation of this dissertation is that the set of PPT operators
and the set of separable operators both form closed convex cones and many problems concerning them, 
including state distinguishability, can be phrased in terms of cone programming,
which is a convex optimization framework that generalizes semidefinite programming.
In general, we do not have a classical polynomial-time algorithm to solve
cone programs and, in fact, optimizing over separable operators is an NP-hard task. 
Nevertheless, cone programming, like semidefinite programming, comes with a rich duality theory,
which can be exploited in order to derive analytical bounds for the problems we are seeking to solve.
From the numerical point of view, we exploit the fact that the particular cone programs
we are interested in can be approximated by efficiently solvable hierarchies of 
semidefinite programs \cite{Doherty02}.

%-------------------------------------------------------------------------------
\section{Summary of the results}
%-------------------------------------------------------------------------------
We prove the following specific results:
\begin{itemize}
\item We obtain an exact formula for the optimal probability of correctly 
discriminating any set of either three or four Bell states via separable 
measurements, when the parties are given a partially entangled pair of qubits as a resource. 
In particular, it is proved that this ancillary pair of qubits must be maximally 
entangled in order for three Bell states to be perfectly discriminated
by separable (or LOCC) measurements, which answers an open question of \cite{Yu14}.
\item We build up on a construction by Yu, Duan, and Ying \cite{Yu12}, and show
the first example of a set of $k < n$ orthogonal maximally entangled states 
in $\complex^{n}\otimes\complex^{n}$ that is not perfectly distinguishable by LOCC,
One takeaway from this is that the local dimension of the local subsystems does not 
play any special role in the nonlocality exhibited by LOCC-indistinguishable sets of 
maximally entangled states. 
The same example serves to exhibit a gap between the power of separable and PPT measurements
for the task of distinguishing maximally entangled states.
\item We provide an easily checkable characterization of when an unextendable 
product set is perfectly discriminated by separable measurements, and we use
this characterization to present an example of an unextendable product set in
$\complex^{4}\otimes\complex^{4}$ that is not perfectly discriminated by 
separable measurements. This resolves an open question raised in
\cite{Duan09}. 
\item We show that every unextendable product set together 
with one extra pure state orthogonal to every member of the unextendable product 
set is not perfectly discriminated by separable measurements.
\end{itemize}

%-------------------------------------------------------------------------------
\section{Overview}
%-------------------------------------------------------------------------------

We assume that the reader is familiar with basic concepts of quantum computation and 
the main target is a researcher in quantum information or a graduate student 
who has taken an introductory course to quantum based on Nielsen and Chuang \cite{Nielsen11}.
Familiarity with more advanced concepts of quantum information theory (based 
on \cite{Watrous15}, for example) would certain help, but it is not necessary.
The same can be said about notions of convex optimization. The syllabus of an introductory 
graduate-level course in convex optimization covers more than it is necessary to grasp
the material of this thesis.
In Chapter \ref{chap:preliminaries} basic notions of quantum information are reviewed. 

Chapter \ref{chap:bipartite-state-discrimination} reviews background material 
on bipartite state discrimination, including a comparison between previous approaches 
to the problem and ours.

In Chapter \ref{chap:programs}, we lay out a general cone programming framework
for bipartite state discrimination and we instantiate it for the particular cases
of separable and PPT measurement. 

In Chapters \ref{chap:mes} and \ref{chap:ups}, we apply the framework described in Chapter 
\ref{chap:programs} to study the distinguishability of sets of maximally entangled states, 
and unextendable product sets, respectively.
These two chapters are independent from each other and they can be read in any order.

In the last chapter we draw conclusions and ask some open questions that may be
of interest for future work. 

The thesis is based on the following papers:
\begin{itemize}

\item[$\bullet$]
 A. Cosentino. 
\textbf{PPT-indistinguishable states via semidefinite programming}. 
\textit{Physical Review A}, 2013.
\cite{Cosentino13}

\item[$\bullet$]
A. Cosentino and V. Russo.
\textbf{Small sets of locally indistinguishable orthogonal maximally entangled states}. 
\textit{Quantum Information \& Computation},  2014.
\cite{Cosentino14}

\item[$\bullet$] 
S. Bandyopadhyay, A. Cosentino, N. Johnston, V. Russo, J. Watrous, and N. Yu. 
\textbf{Limitations on separable measurements by convex optimization}. 
\textit{IEEE Transactions on Information Theory}, 2015.
\cite{Bandyopadhyay15}

\end{itemize}